\chapter{Introduction}
\bigskip

\noindent If anything good can ever be said about the second world war, it might be this: the war effort sparked a massive number of scientific fields.

Though most fields existed prior to the war, after the war they were funded by the public as strategic pieces of the major nations arsenal against future conflagrations. One of the fields in question was that of Management Science (also called Operations Research in military circles, as researchers filled the ranks of planners of war operations). Management Science had started as an industrial field, in movements stemming from Taylor and the origin of the production line by Henry Ford.  That was the first moment in industry in which operations were systematically subject to some of the tools of science: measurement, experimentation, hypothesis-testing, statistics, mathematical optimization, etc.

This is from almost 100 years ago. Today the field has advanced to a great number of nations, and the amount of applications has grown explosively.  Of particular interest to us is the advent of the computer, and of engineering efforts that brought exponential growth in computational power at the hands of individuals.  Whereas, during the war, computations were mostly done by hand, it is not outlandish to say that this original field can be referred to, today, as computational management science.

Applied mathematics and computer science serve simultaneously as a theoretical foundation and the major tool available to the field.  Though this is a doctoral thesis concerning business, in this document one should expect to find the language and nomenclature of mathematical modeling and computer science as our primary and most natural language.

This thesis will present a number of different topics for exploration. Though the range of the topics is large, as it usually is in management science, it is my hope to convince readers of the value of this doctoral thesis brought by three self-contained scientific papers.  The first of which will study the possibility of distributed financial ledgers.

\section{Distributed financial ledgers}

United Nations World Food Programme, `Blockchain Against Hunger: Harnessing Technology In Support Of Syrian Refugees', 30 May 2017, https://www.wfp.org/news/news-release/blockchain-against-hunger-harnessing-technology-support-syrian-refugees, accessed in January 17, 2018.

United Nations World Food Programme, 2016, The Year in Review Report, p.

The first paper lies at the intersection of computer science, record-keeping, banking \& finance, and economic inclusion. The seminal work of \citet{nakamoto2008bitcoin} described the architecture of Bitcoin, a peer-to-peer electronic cash system, also known as cryptocurrency. Bitcoin's currency ledger is public and stored in a blockchain across thousands of computers. Even so, no one is able to spend either somebody else's funds nor to double spend their own funds. In order to be confirmed, each transaction must be both digitally signed by the owner of the money and the funds verified in the blockchain by Bitcoin's miners. One of the interesting parts of Bitcoin are the incentives. On one hand, users have incentive to use Bitcoin because the fees are small, the money transfer is quick, and the currency issuance rate is well known. On the other hand, the miners have incentive to be part of Bitcoin's network because new coins are found every ten minutes.

The impact of Bitcoin in society -- and hence in the companies and the government -- has been increasing every day. People are increasingly using Bitcoin to exchange money and transfer money overseas. Companies are looking into Bitcoin as an alternative to reduce banking fees. The poor may be included in the finance system through Bitcoin. People may hedge their assets against their governments' money issuance and inflation---as in the sad case of Venezuela.

Bitcoin is the first and most famous cryptocurrency, used worldwide and has a highly volatile market cap, as of this writing, of \$ 224 mi. Even so, it faces serious scalability challenges; such as serious quality of service and network congestion when the number of transactions per second is high, and an increase in the transaction fees and uncertain delays in transactions' confirmations.

Note that these problems have been a deliberate decision from the current developers of the ``bitcoin-core'' developers, which believe that is it risky to increase the the blocksize (in which all transactions are stored).  It is not known whether a blocksize, say, of 1Gb, would be feasible to sustain the decentralization of the network. 

Iota is another cryptocurrency, but, instead of using a blockchain, it proposes the use a ``tangle' architecture', which is a different way to store the currency ledger across thousands of computers. Although it has not been confirmed in practice yet, its architecture seems to be significantly more scalable than Bitcoin's blockchain. As we will see, the problem here is exactly the opposite of Bitcoin's. It needs a minimum of transactions per seconds in order to work properly.

Our analysis suggests an architecture for a distributed currency which is inspired in both Bitcoin's blockchain and Iota's tangle in order to solve the scalability problems. While Bitcoin's network saturates when it hits a certain number of transactions per second, Iota's does not work properly with less than a certain number of transactions per second. Our proposed architecture works in both scenarios: low and high number of transactions per second.

We investigate some issues regarding this possibility, namely: (i) cryptographic security and game-theoretical attacks; (ii) scalability; (iii) self-governance of the system; (iv) appropriate incentive system to all participants.


\section{Artificial Intelligence}

%The second paper lies at the intersection of cognitive psychology, computer science, neuroscience, and artificial intelligence.  Sparse Distributed Memory, or SDM for short, is a theoretical mathematical construct that seems to reflect a number of neuroscientific and psychologically plausible characteristics. We implement a RB-Complete\footnote{Ridiculously Buzzword Complete: the model is (i) Open-Source, (ii) Cross-Platform; (iii) highly parallel; (iv) able to execute on CPUs and/or GPUs; (v) it can be run on the `cloud'; etc.} SDM framework that not only shows small discrepancies from theoretical expectations, but also may be of use to other researchers interested in testing their own hypotheses and theories of SDM. The maturity of this work is medium-high: The computer code has been used in a previous Ph.D. Thesis; the code has shown some small discrepancies from theoretical expectations; the code has been ran on a number of different architectures and information-processing devices (e.g., CPUs, GPUs).  What remains to be done is a full description of the systems' architecture and a performance comparison between different setups.
The second paper lies at the intersection of cognitive psychology, computer science, neuroscience, and artificial intelligence.  Sparse Distributed Memory, or SDM for short, is a theoretical mathematical construct that seems to reflect a number of neuroscientific and psychologically plausible characteristics. We implement a RB-Complete\footnote{Ridiculously Buzzword Complete: the model is (i) Open-Source, (ii) Cross-Platform; (iii) highly parallel; (iv) able to execute on CPUs and/or GPUs; (v) it can be run on the `cloud'; etc.} SDM framework that not only shows small discrepancies from theoretical expectations, but also may be of use to other researchers interested in testing their own hypotheses and theories of SDM. The computer code has been used in a previous Ph.D. Thesis; the code has shown some small discrepancies from theoretical expectations; the code has been ran on a number of different architectures and information-processing devices (e.g., CPUs, GPUs).  What remains to be done is a full description of the systems' architecture and a performance comparison between different setups.

The third paper lies at the intersection of Marketing, Diffusion of Technological Innovation, and modeling. The classic Bass model of diffusion of innovation is extended, in order to account for users who, after adopting the innovation for a while, decide to reject it later on (possibly bringing down the number of active users---which is impossible in Bass' original model). Four alternative mathematical models are presented and discussed with the Facebook users dataset. %This is the work that has received most energy so far, and is the most mature of the papers presented herein.  I believe it is almost ready for submission for publication. 

%A fourth paper is still highly immature.  It concerns monetary inflation, and the demonstrations that an approach through fixed effects cannot correct for the real cumulative inflation rates---and may be disastrous when referring to regions undergoing high inflation.  Though part of the mathematical analysis is mature, the work itself is lacking the necessary references to show its relevance:  I still need to review the literature and point out the rather large body of it that is committing a serious error. Because I have not decided whether this work should be part of the thesis, it will be presented in appendix 1.

A note on the overall format of this Thesis Proposal:  many journals, such as Nature, have a series of papers departing from the standard layout of (i) introduction, (ii) literature review, (iii) methodology, (iv) experiments, (v) discussion and (vi) conclusion.  Instead, these journals will generally present (i) introduction, (ii) discussion and conclusion, and---sometimes in smaller fonts---, the materials and methods and results section. Here, content is brought to focus, in detriment to form.  As I believe that the content of the studies should be the major focus of analysis for this proposal, I have also collapsed the introductory and concluding remarks into these few comments.  I hope readers will agree that, at this point, it is of overwhelmingly greater importance to query the scientific content of the proposal, even if in detriment to its overall form.

It is my hope that readers of this proposal will accept the format of self-contained studies and will agree that---if the topics discussed herein are handled with the proper effort and care---the proposed body of work should be deemed worthy of the originality and the scientific significance implicit to the work of a Doctoral Thesis.
