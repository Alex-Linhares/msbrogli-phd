% !TEX root = ../partial-sdm.tex

\chapter{Introduction}

Sparse Distributed Memory (SDM) \citep{Kanerva1988} is a mathematical model of long-term memory that has a number of neuroscientific and psychologically plausible dynamics. This model may be applied in all sort of applications because of its incredible ability to closely reflect the human capacity to remember past experiences from clues of the present. For instance, when one is walking on a dark alley and is afraid of something, one cannot explain where one's fear come from. One just feels it. We may interpret this situation as clues of the present --- a dark alley; a giant metropolitan area; people going on about their lives mostly indifferent from each other; constrained routes ahead and behind you; etc. --- recalling past experiences from memory and thus generating the feeling. Our memory is able to make a parallel between previous experiences and the clues. Although one has never been in the exactly same situation, one's brain involuntarily makes an analogy and recognizes the possibility of danger. This flexibility into mapping one situation in another is an important human feature which is hard to replicate in computers.

SDM has been applied in many different fields, like pattern recognition \citep{norman2003modeling, rao1995natural}, noise reduction \citep{Meng2009}, handwriting recognition \citep{fan1997genetic}, robot automation \citep{Rajesh1998, mendes2008robot}, and so forth. \citet{Linhares2011} has showed that SDM respects the limits of short-term memory discussed by \citet{Miller1995} and \citet{Cowan2011}. Despite all those applications, there is not a reference implementation which would allow one to replicate the results published in a paper, to check the source code for details, and to improve it. Thus, even though intriguing results have been achieved using SDM, it requires great effort from researchers to build on top of previous work.

It is our belief that such a tool could bring orders of magnitude more researchers and attention if they were able to use the model, at zero cost, with an easy to use high-level language such as python, in an intuitive platform such as juypyter notebooks. Neuroscientists interested in long-term memory storage should not have to worry about high-bandwidth vector parallel computation.  This new tool provides a ready to use system in which experiments can be executed almost as soon as they are designed and it may accelerate research \citep{shen2014interactive}.

Our motivation was our own effort to run our models. As there is no reference implementation, we had to implement our own and run several simulations to ensure that our implementation was correct and bug free. Thus, we had to deviate from our main goal --- which was to test our hypothesis and explore the `idea space' --- and to focus in the implementation itself. Furthermore, new members in our research group had to go through different source codes developed by former members.

Extensions of SDM have been used in many applications. For example, \citet{Snaider2011} extended SDM to efficiently store sequences of vectors and trees.  \citet{Rajesh1998} used a modified SDM in an autonomous robot. \citet{Meng2009} modified SDM to clean patterns from noisy inputs. \citet{fan1997genetic} extended SDM with genetic algorithms. \citet{chada2016you} extended SDM creating the Rotational Sparse Distributed Memory (RSDM), which is used to modeling network motifs, dynamic flexibility, and hierarchical organization, all results from neuroscience literature.

The main contribution of this work is a reference implementation which yields (i) orders of magnitude gains in performance, (ii) has several backends and operations, (iii) has been validated against the mathematical model, (iv) is cross-platform, and (v) is easily extended to test new research ideas. Our reference implementation may, hopefully, accelerate research into the model's dynamics and make it easier for readers to replicate any previous results and easily understand the source-code of the model.  Moreover, it is compatible with jupyter notebook and researchers may share their notebooks possibly accelerating the advances in their fields \citep{shen2014interactive}.

Other contributions have also been introduced, which include (i) a noise filtering approach, (ii) a supervised classification algorithm, (iii) and a reinforcement learning algorithm, all of them using only the original SDM proposed by Kanerva, i.e., with no additional mechanisms, algorithms, data structures, etc. Although some of these applications have already been explored in previous work \citep{Meng2009, fan1997genetic, rao1995natural}, all of them have done some adapting of SDM to their problems, and none of them have used just the ideas introduced by Kanerva. We have presented different approaches with no adaptations whatsoever.

Finally, I have striven to provide a visual tour of the theory and application of SDM: whenever possible, detailed figures should tell the story --- or at least do the heavy lifting. In this study, we will see an anomaly in one of Kanerva's predictions, which I believe is related to SDM capacity. We will see tests of a generalized reading operation proposed by Physics Professor Paulo Murilo (personal communication).  We will see what happens when neurons --- and
all their information --- is simply and suddenly lost.  We will see whether information-theory can improve some of Kanerva's ideas.  From (basic) noise filtering to learning to play tic-tac-toe, we will review the entirety of Dr. Pentti Kanerva's proposal.

This time, however, it will be running on a computer.

\chapter{Notation}

\begin{tabular}{cp{\textwidth}}
  $n$ & Number of dimensions, i.e., $n=1,000$. \\
  $N$ & Size of the binary space, $|\{0, 1\}^n| = 2^n$. \\
  $N'$ & Number of hard-locations samples from $\{0, 1\}^n$. Its typical value is 1,000,000, as suggested by \citet{Kanerva1988}. \\
  $H$ & Same as $N'$. \\
  $r$ & Access radius, i.e., when $n=1,000$ and $N'=1,000,000$, its typical value is $451$. This value is calculated to activate, on average, one thousandth of $N'$. \\
  $\eta$ & A bitstring, usually a datum. \\
  $\eta_x$ & A clue $x$ bits away from $\eta$, i.e., $\text{dist}(\eta, \eta_x) = x$. \\
  $\xi$ & A bitstring, usually an address. \\
  $\text{dist}(x, y)$ & Hamming distance between $x$ and $y$. \\
  $\text{d}(x, y)$ & Same as $\text{dist}(x, y)$.
\end{tabular}\\

\chapter{Sparse Distributed Memory}
% !TEX root = ../partial-sdm.tex

Sparse Distributed Memory (SDM) is a mathematical model for cognitive memory published by \citet{Kanerva1988}. It introduces many interesting mathematical properties of $n$-dimensional binary space that, in a memory model, are psychologically plausible.  Most notable among these are the tip-of-the-tongue phenomenon, conformity to Miller's magic number \citep{Linhares2011} and robustness against loss of neurons.

The data and address space belong to binary space and are represented by a sequence of bits, called bitstrings. The distance between two bitstrings is calculated using the Hamming distance. It is defined for two bitstrings of equal length as the number of positions at which the bits are different. For example, $00110_{b}$ and $01100_{b}$ are bitstrings of length 5 and their Hamming distance is 2. One has to be careful when thinking intuitively about distance in SDM because the Hamming distance does not have the same properties of, say, the Euclidean distance.

\begin{figure}[p!]
  \centering
  \subfloat[$Q_3$]{\includegraphics[width=2.2in]{./images02/new-images/qn3.png}}
  \subfloat[$Q_7$]{\includegraphics[width=2.2in]{./images02/new-images/qn7.png}}

  \subfloat[$Q_{10}$]{\includegraphics[width=4.0in]{./images02/new-images/qn10.png}}
  \caption{Here we have $Q_n$, for $n \in$ \{3, 7, 10\}. Each node represents a possible bitstring in $\{0,1\}^n$, and two nodes are linked if the bitstrings differ by a single dimension.  A number of observations can be made here. First, the number of nodes grows as $2^n$ as $n$ grows; which makes the  space intractable as $n>>20$. Another interesting observation, better seen in the figures below, is that most of the space lies `at the center', at a distance of around 500 from any given vantage point.\label{hypercubes}}
\end{figure}



The graph composed of $\{0,1\}^n$ nodes and links between nodes $iff$ their Hamming distance is one is called the \emph{hypercube graph}, or $Q_n$, as in Figure \ref{hypercubes}.  Though Kanerva has derived many combinatorial properties of the space, I believe that this is a aesthetically appealing object on its own, and beautiful results can be found in the graph-theoretical literature. A good survey is found in \citet{harary1988survey}.


Unlike traditional memory used by computers, SDM performs read and write operations in a multitude of addresses, also called neurons.  That is, the data is not written, or it is not read in a single address spot, but in many addresses. These are called activated addresses, or activated neurons.

The activation of addresses takes place according to their distances from the datum. Suppose one is writing datum $\eta$ at address $\xi$, then all addresses inside a circle with center $\xi$ and radius $r$ are activated. So, $\eta$ will be stored in all these activated addresses, which are around address $\xi$, such as in Figure \ref{fig-addresses-inside-access-radius}.  An address $\xi'$ is inside the circle if its hamming distance to the center $\xi$ is less than or equal to the radius $r$, i.e. $distance(\xi,\xi')\leq r$.

\begin{figure}[!htb]
\centering\includegraphics[scale=0.75]{./images02/p_circle_r.pdf}

\caption{Activated addresses inside access \protect \\
radius $r$ around center address.\label{fig-addresses-inside-access-radius}}
\end{figure}



Every time write or read in SDM memory activates a number of addresses with close distance.  The data is written in these activated addresses or read from them.  These issues will be addressed in due detail further on, but a major difference from a traditional computer memory is that the data are always stored and retrieved in a multitude of addresses. This way SDM memory has robustness against loss of addresses (e.g., death of a neuron).

In traditional memory, each datum is stored in an address and every look up of a specific datum requires a search through the memory. In spite of computer scientists having developed beautiful algorithms to perform fast searches, almost all of them do a precise search. That is, if you have an imprecise clue of what you need, these algorithms will simply fail.

In SDM, the data space is the same as the address space, which amounts to a vectorial, binary space, that is, a $\{0,1\}^{n}$ space. This way, the addresses where the data will be written are the same as the data themselves. For example, the datum $\eta=00101_{b}\in\{0,1\}^{5}$ will be written to the address $\xi=\eta=00101_{b}$. If one chooses a radius of 1, the SDM will activate all addresses one bit away or less from the center address. So, the datum $00101_{b}$ will be written to the addresses $00101_{b}$, $10101_{b}$, $01101_{b}$, $00001_{b}$, $00111_{b}$, and $00100_{b}$.

In this case, when one needs to retrieve the data, one could have an imprecise cue at most one bit away from $\eta$, since all addresses one bit away have $\eta$ stored in themselves.  Extending this train of thought for larger dimensions and radius, exponential numbers of addresses are activated and one can see why SDM is a distributed memory.

When reading a cue $\eta_{x}$ that is $x$ bits away of $\eta$, the cue shares many addresses with $\eta$. The number of shared addresses decreases as the cue's distance to $\eta$ increases, in other words, as $x$ increases. This is shown in Figure \ref{fig-shared-addresses}.  The target datum $\eta$ was written in all shared addresses, thus they will bias the read output in the direction of $\eta$. If the cue is sufficiently near the target datum $\eta$, the read output will be closer to $\eta$ than $\eta_{x}$ was. Repeating the read operation increasingly gets results closer to $\eta$, until it is exactly the same. So, it may be necessary to perform more than one read operation in order to converge to the target data $\eta$.

\begin{figure}[!htb]
\centering\includegraphics[scale=0.75]{./images02/p1_inter_p2.pdf}

\caption{Shared addresses between the \protect \\
target datum $\eta$ and the cue $\eta_{x}$. \label{fig-shared-addresses}}
\end{figure}


The addresses of the $\{0,1\}^{n}$ space grows exponentially with the number of dimensions $n$, i.e. $N=2^{n}$. For $n=100$ we have $N\approx10^{30}$, which is incredibly large when related to a computer memory. Furthermore, \citet{Kanerva1988} suggests $n$ between 100 and 10,000. Recently he has postulated 10,000 as a desirable minimum $N$ (personal communication). To solve the feasibility problem of implementing this memory, Kanerva made a random sample of $\{0,1\}^{n}$, in his work, having $N'$ elements. All these addresses in the sample are called hard-locations. Other elements of $\{0,1\}^{n}$, not in $N'$, are called virtual neurons. This is represented in Figure \ref{fig-hardlocations}.  All properties of read and write operations presented before remain valid, but limited to hard-locations. Kanerva suggests taking a sample of about one million hard-locations.

Using this sample of binary space, our data space does not exist completely.  That is, the binary space has $2^{n}$ addresses, but the memory is far away from having these addresses available. In fact, only a fraction of this vectorial space is actually instantiated. Following Kanerva's suggestion of one million hard-locations, for $n=100$, only $100\cdot10^{6}/2^{100}=7\cdot10^{-23}$ percent of the whole space exists, and for $n=1,000$ only $100\cdot10^{6}/2^{1000}=7\cdot10^{-294}$ percent.

Kanerva also suggests the selection of a radius that will activate, on average, one one thousandth of the sample, which is 1,000 hard-locations for a sample of one million addresses. In order to achieve his suggestion, a 1,000-dimension memory uses an access radius $r=451$, and a 256-dimensional memory, $r=103$. We think that a 256-dimensional memory may be important because it presents conformity to Miller's magic number \citep{Linhares2011}.

\begin{figure}[!htb]
\centering\includegraphics[scale=0.75]{./images02/hardlocations.pdf}

\caption{Hard-locations randomly sampled from binary space.\label{fig-hardlocations}}
\end{figure}


Since a cue $\eta_{x}$ near the target bitstring $\eta$ shares many hard-locations with $\eta$, SDM can retrieve data from imprecise cues. Despite this feature, it is very important to know how imprecise this cue could be while still giving accurate results. What is the maximum distance from our cue to the original data that still retrieves the right answer? An interesting approach is to perform a read operation with a cue $\eta_{x}$, that is $x$ bits away from the target $\eta$.  Then measure the distance from the read output and $\eta$. If this distance is smaller than $x$ we are converging. Convergence is simple to handle, just read again and again, until it converges to the target $\eta$. If this distance is greater than $x$ we are diverging. Finally, if this distance equals $x$ we are in a tip-of-the-tongue process.  A tip-of-the-tongue psychologically happens when you know that you know, but you can't say what exactly it is. In SDM mathematical model, a tip-of-the-tongue process takes infinite time to converge. \citet{Kanerva1988} called this $x$ distance, where the read's output averages $x$, the critical distance. Intuitively, it is the distance from which smaller distances converge and greater distances diverge. In Figure \ref{fig-p1-p2-iterative-read}, the circle has radius equal to the critical distance and every $\eta_{x}$ inside the circle should converge.  The figure also shows a convergence in four readings.

\begin{figure}[!htb]
\centering\includegraphics[scale=0.75]{./images02/p1_p2_iter_read.pdf}

\caption{In this example, four iterative readings were\protect \\
required to converge from $\eta_{x}$ to $\eta$.\label{fig-p1-p2-iterative-read}}
\end{figure}


The $\{0,1\}^{n}$ space has $N=2^{n}$ locations from which we instantiate $N'$ samples. Each location in our sample is called a hard-location.  On these hard-locations we do operations of read and write. One of the insights of SDM is exactly the way we read and write: using data as addresses in a distributed fashion. Each datum $\eta$ is written in every activated hard-location inside the access radius centered on the address, that equals datum, $\xi=\eta$. Kanerva suggested using an access radius $r$ having about one one thousandth of $N'$.  As an imprecise cue $\eta_{x}$ shares hard-locations with the target bitstring $\eta,$ it is possible to retrieve $\eta$ correctly. (Actually, probably more than one read is necessary to retrieve exactly $\eta.)$.  Moreover, if some neurons are lost, only a fraction of the datum is lost and it is possible that the memory can still retrieve the right datum.

A random bitstring is generated with equal probability of $0$'s and $1$'s in each bit. One can readily see that the average distance between two random bitstrings has binomial distribution with mean $n/2$ and standard deviation $\sqrt{n/4}$. For a large $n$, most of the space lies close to the mean and has fewer shared hard-locations.  As two bitstrings with distance far from $n/2$ are very improbable, \citet{Kanerva1988} defined that two bitstrings are orthogonal when their distance is $n/2$.

The write operation needs to store, for each dimension bit which happened more ($0$'s or $1$'s). This way, each hard-location has $n$ counters, one for each dimension. The counter is incremented for each bit $1$ and decremented for each bit $0$. Thus, if the counter is positive, there have been more $1$'s than $0$'s, if the counter is negative, there have been more $0$'s than $1$'s, and if the counter is zero, there have been an equal number of $1$'s and $0$'s. Table \ref{tab:write operation} shows an example of a write operation being performed in a 7-dimensional memory.

\begin{table}
\begin{tabular}{c|c|c|c|c|c|c|c|}
\cline{2-8}
$\eta$ & 0 & 1 & 1 & 0 & 1 & 0 & 0\tabularnewline
\cline{2-8}
$\xi_{\textit{before}}$ & 6 & -3 & 12 & -1 & 0 & 2 & 4\tabularnewline
\cline{2-8}
\multicolumn{1}{c}{} & \multicolumn{1}{c}{\textcolor{red}{\small{}$\Downarrow$ -1}} & \multicolumn{1}{c}{\textcolor{red}{\small{}$\Downarrow$ +1}} & \multicolumn{1}{c}{\textcolor{red}{\small{}$\Downarrow$ +1}} & \multicolumn{1}{c}{\textcolor{red}{\small{}$\Downarrow$ -1}} & \multicolumn{1}{c}{\textcolor{red}{\small{}$\Downarrow$ +1}} & \multicolumn{1}{c}{\textcolor{red}{\small{}$\Downarrow$ -1}} & \multicolumn{1}{c}{\textcolor{red}{\small{}$\Downarrow$ -1}}\tabularnewline
\cline{2-8}
$\xi_{\textit{\textit{after}}}$ & \textbf{5} & \textbf{-2} & \textbf{13} & \textbf{-2} & \textbf{1} & \textbf{1} & \textbf{3}\tabularnewline
\cline{2-8}
\end{tabular}

\caption{Write operation example in a 7-dimensional memory of data $\eta$
being written to $\xi$, one of the activated addresses.\label{tab:write operation}}


\end{table}


The read is performed polling each activated hard-location and statistically choosing the most written bit for each dimension. It consists of adding all $n$ counters from the activated hard-locations and, for each bit, choosing bit 1 if the counter is positive, choose bit 0 if the counter if negative, and randomly choose bit 0 or 1 if the counter is zero.


\section{Neurons as pointers}

One interesting view is that neurons in SDM work like pointers. As we write bitstrings in memory, the hard-locations' counters are updated and some bits are flipped. Thus, the activated hard-locations do not necessarily point individually to the bitstring that activated it, but together they point correctly. In other words, the read operation depends on many hard-locations to be successful. This effect is represented in Figure \ref{fig-p1-pointers}: where all hard-locations inside the circle are activated and they, individually, do not point to $\eta$.  But, like vectors, adding them up points to $\eta$. If another datum $\nu$ is written into the memory near $\eta$, the shared hard-locations will have information from both of them and would not point to either.  All hard-locations outside of the circle are also pointing somewhere (possibly other data points). This is not shown, however, in order to keep the picture clean and easily understandable.

\begin{figure}[!htb]
\centering\includegraphics[scale=0.75]{./images02/p1_after_write.pdf}

\caption{Hard-locations pointing, approximately, to the target bitstring.\label{fig-p1-pointers}}
\end{figure}



\section{Concepts}

Although Kanerva does not mention concepts directly in his book \citep{Kanerva1988}, the author's interpretation is that each bitstring may be mapped to a concept. Thus, unrelated concepts are orthogonal and concepts could be linked through a bitstring near both of them. For example, ``beauty'' and ``woman'' have distance $n/2$, but a bitstring that means ``beautiful woman'' could have distance $n/4$ to both of them. As a bitstring with distance $n/4$ is very improbable, it is linking those concepts together. \citet{Linhares2011} approached this concept via ``chunking through averaging''.

Due to the distribution of hard-locations between two random bitstrings, the vast majority of concepts is orthogonal to all others. Consider a non-scientific survey during a cognitive science seminar, where students asked to mention ideas unrelated to the course brought up terms like birthdays, boots, dinosaurs, fever, executive order, x-rays, and so on. Not only are the items unrelated to cognitive science, the topic of the seminar, but they are also unrelated to each other.

For any two memory items, one can readily find a stream of thought relating two such items (``Darwin gave dinosaurs the boot''; ``she ran a fever on her birthday''; ``isn't it time for the Supreme Court to x-ray that executive order?'', ... and so forth). Robert French presents an intriguing example in which one suddenly creates a representation linking the otherwise unrelated concepts of ``coffee cups'' and ``old elephants'' \citep{French1997}.

This mapping from concepts to bitstrings brings us two main questions: (i) Suppose we have a bitstring that is linking two major concepts.  How do we know which concepts are linked together? (ii) From a concept bitstring how can we list all concepts that are somehow linked to it? This second question is called the problem of spreading activation.


\section{Read operation}

In his work, Kanerva proposed and analyzed a read algorithm called here Kanerva's read. His read takes all activated hard-locations counters and sum them. The resulting bitstring has bit $1$ where the result is positive, bit $0$ where the result is negative, and a random bit where the result is zero. In a word, each bit is chosen according to all written bitstrings in all hard-locations, being equal to the bit more appeared. Table \ref{tab:kanerva-read} shows an example of Kanerva's read result bitstring.

Daniel Chada, one member of our research group, proposed another way to read in SDM, in this work called Chada's read. Instead of summing all hard-location counters, each hard-location evaluates its resulting bitstring individually. Then, all resulting bitstrings are summed again, and the same rule as Kanerva applies. Table \ref{tab:chada-read} shows an example of Chada's read result bitstring. The counter's values are normalized to 1, for positive ones, or -1, for negative ones, and the original values are the same as in Table \ref{tab:kanerva-read}.

The main change between Kanerva's read and Chada's read is that, in the former, a hard-location that has more bitstrings written has a greater weight in the decision of each bit. In the latter, all hard-locations have the same weight, because they can contribute to the sum with only one bitstring.

A member of my Master's committee, professor Paulo Murilo, has proposed a generalized reading operation (personal communication), which covers both Kanerva's and Chada's read --- and opens a new venue of potential discoveries. He proposed summing all hard-location counters raised to the power of $z$ while holding the original sign of the counter (positive or negative). Thus, Kanerva's read would be the same as $z=1$, while Chada's would be the same as $z=0$. Hence, we will here explore how SDM would behave with other values of $z$, such as 0.5, 2, and 3.

\begin{table}
\begin{minipage}[t]{0.5\columnwidth}%
\subfloat[Kanerva's read example\label{tab:kanerva-read}]{%
\begin{tabular}{c|c|c|c|c|c|}
\cline{2-6}
$\xi_{1}$ & -2 & 12 & 4 & 0 & -3\tabularnewline
\cline{2-6}
$\xi_{2}$ & -5 & -4 & 2 & 8 & -2\tabularnewline
\cline{2-6}
$\xi_{3}$ & -1 & 0 & -1 & -2 & -1\tabularnewline
\cline{2-6}
$\xi_{4}$ & 3 & 2 & -1 & 3 & 1\tabularnewline
\hline
\multicolumn{1}{|c|}{\textbf{$\sum$}} & \textbf{-5} & \textbf{10} & \textbf{4} & \textbf{3} & \textbf{-5}\tabularnewline
\hline
\multicolumn{1}{c}{} & \multicolumn{1}{c}{\textcolor{red}{\small{}$\Downarrow$}} & \multicolumn{1}{c}{\textcolor{red}{\small{}$\Downarrow$}} & \multicolumn{1}{c}{\textcolor{red}{\small{}$\Downarrow$}} & \multicolumn{1}{c}{\textcolor{red}{\small{}$\Downarrow$}} & \multicolumn{1}{c}{\textcolor{red}{\small{}$\Downarrow$}}\tabularnewline
\cline{2-6}
 & 0 & 1 & 1 & 1 & 0\tabularnewline
\cline{2-6}
\end{tabular}

}%
\end{minipage}%
\begin{minipage}[t]{0.5\columnwidth}%
\subfloat[Chada's read example\label{tab:chada-read}]{%
\begin{tabular}{c|c|c|c|c|c|}
\cline{2-6}
$\xi_{1}$ & -1 & 1 & 1 & \cellcolor{lightgray}1 & -3\tabularnewline
\cline{2-6}
$\xi_{2}$ & -1 & -1 & 1 & 1 & -1\tabularnewline
\cline{2-6}
$\xi_{3}$ & -1 & \cellcolor{lightgray}1 & -1 & -1 & -1\tabularnewline
\cline{2-6}
$\xi_{4}$ & 1 & 1 & -1 & -1 & 1\tabularnewline
\hline
\multicolumn{1}{|c|}{\textbf{$\sum$}} & \textbf{-2} & \textbf{1} & \textbf{0} & \textbf{0} & \textbf{-2}\tabularnewline
\hline
\multicolumn{1}{c}{} & \multicolumn{1}{c}{\textcolor{red}{\small{}$\Downarrow$}} & \multicolumn{1}{c}{\textcolor{red}{\small{}$\Downarrow$}} & \multicolumn{1}{c}{\textcolor{red}{\small{}$\Downarrow$}} & \multicolumn{1}{c}{\textcolor{red}{\small{}$\Downarrow$}} & \multicolumn{1}{c}{\textcolor{red}{\small{}$\Downarrow$}}\tabularnewline
\cline{2-6}
 & 0 & 1 & \cellcolor{lightgray}1 & \cellcolor{lightgray}1 & 0\tabularnewline
\cline{2-6}
\end{tabular}

}%
\end{minipage}\caption{Comparison of Kanerva's read and Chada's read. Each $\xi_{i}$ is
an activated hard-location and the values come from their counters.
Gray cells' value is obtained randomly with probability 50\%.\label{tab:read-operation}}
\end{table}


\section{Critical Distance}

Kanerva describes the critical distance as the threshold of convergence of a sequence of read words. It is ``the distance beyond which divergence is more likely than convergence''\citep{Kanerva1988}. Furthermore, Kanerva explains that ``a very good estimate of the critical distance can be obtained by finding the distance at which the arithmetic mean of the new distance to the target equals the old distance to the target''\citep{Kanerva1988}.  In other words, the critical distance can be equated as the edge to our memory, the limit of human recollection.

In his book, Kanerva analyzed a specific situation with $n=1000$ ($N=2^{1000}$), 10 million hard-locations, an access-radius of 451 (within 1000 hard-locations in each circle) and 10 thousand writes of random bitstrings in the memory. As computer resources were very poor those days, Kanerva couldn't make a more generic analysis.

Starting from the premise of SDM as a faithful model of human short-term memory, a better understanding of the critical distance may shed light on our understanding of the thresholds that bind our own memory.


\begin{figure}[!htb]
\centering

\subfloat[Kanerva's original model]{\includegraphics[width=3.1in]{./images02/new-images/kanerva-read.png}}

\subfloat[Chada's read]{\includegraphics[width=3.1in]{./images02/new-images/chada-read_z_0.png}}
\caption{How far, in hamming distance, is a read item from the original stored item? Kanerva demonstrated that, after a small number of iterative readings (6 here), a critical distance behavior emerges. Items read at close distance converge rapidly; whereas farther items do not converge. Most striking is the point in which the system displays the tip-of-tongue behavior. Described by psychological moments when some features of the item are prominent in one's thoughts, yet the item still cannot be recalled (but an additional cue makes convergence `immediate'). Mathematically, this is the precise distance in which, despite having a relatively high number of cues (correct bits) about the desired item, the time to convergence is infinite.   Heatmap colors display the hamming distance the associative memory is able to cleanly converge to---or not.   In the $x$-axis, the distance from the desired item is displayed. In the $y$-axis, we display the read operation's behavior as the number of items registered in the memory grows.  These graphs are computing intensive, yet they can be easily tested by readers in our provided jupyter notebooks. Note the different scales.}
\label{fig:crit-dist-10k-writes}

\end{figure}

Figure \ref{fig:crit-dist-10k-writes} compares the critical distance behavior under different scenarios.  This replicates our previous results \citet{Brogliato2011} and \citet{brogliato2014sparse} and is a first part of the process of framework validation, to which we throw our attention next.



\chapter{Framework Architecture}
% !TEX root = ../partial-sdm.tex

The framework implements the basic operations in a Sparse Distributed Memory which may be used to create more complex operations. It is developed in C language and the OpenCL parallel framework --- which may be loaded in many platforms and programming languages --- with a wrapper in Python. The Python module makes it easy to create and execute simulations in a Sparse Distributed Memory and works properly in Jupyter Notebook \citep{kluyver2016jupyter}. It works in both Python 2 and Python 3.

We split the SDM memory in two parts: the hard-location addresses and the hard-location counters. Thus, the addresses (bitstrings) of the hard locations are stored in one array, while their counters in another. This makes possible to create multiple SDMs using the same address space, which would save computational effort to scan a bitstring in all the SDMs --- since they share the same address space, the activated hard locations will be the same in all of them. As the slowest part of reading and writing operations is scanning the address space, the performance benefits are significant.

Each part may be stored either in the RAM memory or in a file. The RAM memory is interesting for quick experiments, automated tests, and others scenarios in which the SDM may be lost, while the file is interesting for a long-term SDM, like creating an SDM file with 10,000 random writes, which will be copied over and over to run multiple experiments. The file may also be sent to another researcher or may be published within the paper to let others run their own checks and verify the results. In summary, the framework fits many different uses and necessities.

Let a SDM memory with $N$ dimensions and $H$ hard locations. Then, in a 64-bit computer, the array of hard-location addresses will use $H \cdot 8 \cdot \lceil N/64 \rceil$ bytes of memory, and there will be $H \cdot N$ hard-location counters. For example, in a SDM memory with 1,000 dimensions and 1,000,000 hard locations, using 32-bit integers for the counters, the array of addresses will use 122MB of memory and the counters will use 3.8 GB of memory.

Basic operations were grouped in four sets: (i) for bitstrings, (ii) for addresses, (iii) for counters, and (iv) for memories (SDMs). Operations include creating new bitstrings, flipping bits, generating a bitstring with a specific distance from a given bitstring, scanning the address space using different algorithms, writing a bitstring to a counter, writing in an SDM, reading from an SDM, and iteratively reading from an SDM until convergence.


\section{Bitstring}

Bitstrings are the main structure of SDM. The addresses are represented in bitstrings, as well as the data. A bitstring is stored as an array of integers. Each integer may be 16-bit, 32-bit, or 64-bit long, depending on the configuration. By default, each integer is 64-bit long.

For instance, a 1,000-bit bitstring will have $\lceil 1000/64 \rceil = 16$ integers. These integers will have a total of $16 \cdot 64 = 1,024$ bits. The remaining 24 bits are always zero, so they do not affect the result of any operation. The memory usage efficiency is $1 - 24/1024 = 97.65\%$. Bitstrings store neither how many bits they have nor the array length. These pieces of information are only stored in the address space.


\subsection{The distance between two bitstrings}

The distance between two bitstrings is calculated by the hamming distance, which is the number of different bits between them. It is calculated counting the number of ones in the exclusive or (xor) between the bitstrings, i.e., $d(x, y) = \text{number of ones in } x \oplus y$.

There are several algorithms to calculate the number of ones \citep{warren2013hacker}, but the performance depends on the processor. So, we have implemented three different algorithms and one may be selected through compiling flags. The default algorithm is to use a built-in \_\_popcnt() instruction from the compiler.

There is also the naive algorithm, which really counts the number of ones checking bit by bit. It is available only for testing purposes and should never be used.

The other algorithm available is the lookup. It pre-calculates a table with the number of ones of all possible 16-bit integers. This table is accessed a few times to calculate the number of ones of a 64-bit integer, i.e., to calculate the distance between two bitstrings, it sums the distance of each 16-bit part of the bitstrings, i.e., $d(x[0:63], y[0:63]) = d(x[0:15], y[0:15]) + d(x[16:31], y[16:31]) + d(x[32:47], y[32:47]) + d(x[48:63], y[48:63])$ where $x[i:i+15]$ and $y[i, i+15]$ are the 16-bit integers formed by the bits between $i$ and $i+15$ of $x$ and $y$, respectivelly. Each 16-bit distance is calculated through a single table access. As each distance is calculated in O(1), this algorithm runs in O($\lceil bits/16 \rceil$). This table uses 65MB of RAM. One may change the table from 16-bit integers to 32-bit integers, which would halve the number of accesses at the expense of 4GB of RAM (instead of 65MB).


\section{Address space}

An address space is a fixed collection of bitstrings, and each bitstring represents a hard-location address. They store the number of bitstrings, as well as the number of bits, number of integers per bitstring, and the number of remaining bits.

Bitstrings are stored in a contiguous array of 64-bit integers, as shown in Figure \ref{tab:hl-addresses-detail}. Hence, basic pointer arithmetic provides us with performance improvements in their access, as processors realize fetches of contiguous chunks of memory  \citep{pai2004linux}.

\begin{figure}
\centering
\begin{tikzpicture}[
mycell/.style={draw, minimum size=7mm},
matrixA/.style={matrix of nodes,
    nodes={mycell, anchor=center},
    column sep=-\pgflinewidth,
    row sep=-\pgflinewidth,
    },
matrixB/.style={matrix of nodes,
    nodes={mycell, anchor=center},
    column sep=-\pgflinewidth,
    row sep=-\pgflinewidth,
}]

\matrix[matrixA] (A) { addr$_1$ & addr$_2$ & addr$_3$ & $\cdots$ & addr$_H$ \\ };

\matrix[matrixB, below=of A] (B) {
addr$_{k, 1}$ & addr$_{k, 2}$ & addr$_{k, 3}$ & $\cdots$ & addr$_{k, 8 \cdot \lceil N/64 \rceil}$ \\
};

\draw[dashed] (A-1-1.south west)--(B-1-1.north west);
\draw[dashed] (A-1-1.south east)--(B-1-5.north east);
\draw [
	thick,
    decoration={
        brace,
        mirror,
		amplitude=0.2cm,
        raise=0.2cm
    },
    decorate
] (B-1-1.south west) -- (B-1-5.south east)
node [pos=0.5,anchor=north,yshift=-0.5cm] {N bits};

\end{tikzpicture}

\caption{Address space's bitstrings are stored in a contiguous array. In a 64-bit computer, each bitstring is stored in a sub-array of 64-bit integers, with length $8 \cdot \lceil N/64 \rceil$.\label{tab:hl-addresses-detail}}
\end{figure}

The scan for activated hard locations is performed in an address space. It returns the indexes of the bitstrings which were inside the circle (and their distances). Then, each operation uses these pieces of information in a different way.

\subsection{Scanning for activated hard locations}

Scanning for the activated hard locations is a problem similar to well-known problems in computational geometry called ``range reporting in higher dimensions''. In this case, none of the known algorithms is able to solve our problem faster than $O(H)$. The algorithm which seems to best fit in our problem consumes $O(H)$ space and runs in $O(\log^n(H))$ \citep{chazelle1988functional}, which is really slower than $O(H)$ when, for instance, $H=1,000,000$ and $n=1,000$. For a review of the range reporting algorithms, see \citet{chan2011orthogonal}.

In 2014, there was published a solution to fast search in hamming space which seems applicable to our problem \citet{norouzi2014fast}. It provides a fast search when $r/n < 0.11$ or $r/n < 0.06$, where $r$ is the radius and $n$ is the number of bits. But, in our case, for a 1,000 bits SDM, $r/n = 0.451$, which changes the runtime to $O(H^{0.993})$. This is really close to $O(H)$, but with a larger constant. Unfortunately, $O(H)$ is still faster.

It is intriguing that none of those algorithms is able to solve our scanning problem. The idea behind those computational geometry algorithms is roughly to split the search space in half each step, which would take $O(\log(H))$ to go through the whole space. But this approach does not work because of the high number of dimensions (i.e., 1,000) and because the hard locations' addresses are randomly sampled from the $\{0, 1\}^n$ space. Although each addresses' bit itself splits the hard locations in half, it does not split the search space in half since both halves still must be covered by the algorithm. For instance, let's say we have $n=1,000$ dimensions with $H=1,000,000$ hard locations, and we are scanning within a circle with radius $r=451$, then after checking the first bit we have two cases: (i) for the half with the same first bit, we must keep scanning with radius 451; and (ii) for the half with a different first bit, we must keep scanning with radius 450. Hence, the search space has not been split in half because both halves have been covered (and one of them should have been skipped).

Finally, as our best approach is to scan through all hard locations, we may distribute the scan into many tasks which will be executed independently. The tasks may be executed in different processes, threads, or even computers. They may also run in the CPU or in the GPU. In this case, we may take into account both the time required to distribute the tasks and the time to receive their results.

The framework implements three main scanner algorithms: linear scanner, thread scanner, and OpenCL scanner. The linear scanner runs in a single core, is the slowest one, and was developed only for testing purposes; the thread scanner runs at the CPU in multiple threads sharing memory (and our recommendation is to use the number of threads equals to twice the number of CPU cores); and the OpenCL scanner runs in multiple GPU cores and support multiple devices. The speed of a scan depends on the CPU and GPU devices, thus the best approach to choose which scanner is best for one's setup is to run a benchmark.

The OpenCL must be initialized, which just copies the address space's bitstrings to the GPU's memory. Then, many scans may be executed with no necessity to upload the bitstrings again. The OpenCL scanner supports running into multiple devices.

\subsection{OpenCL kernel}

There are 8 OpenCL kernels which explore differently the GPU architecture to improve performance. It is necessary because there are several GPU microarchitecture and a single kernel will never be optimal for all of them. In simplified form, OpenCL splits the tasks into workgroups which, in turn, split their part of the task into workers. The works are like threads in a computer. OpenCL specifies four levels of memory hierarchy for the GPUs: global memory, read-only memory, local memory, and private memory. The global memory and read-only memory are accessible by all workgroups, while each workgroup has its own local memory, accessible by its workers. Finally, each worker has its own private memory. The number of workers per workgroup is defined by user and must be multiple of the number of tasks.

All 8 kernels do the same thing: calculate the exclusive OR (XOR) between two 64-bit integers and count the number of bits one in the result. They just do it with different approaches. For instance, \lstinline{single_scan0} calculates one distance between bitstrings per worker (Listing \ref{lst:single_scan0}); while \lstinline{single_scan2} uses a whole workgroup to calculate each distance, distributing each element of the 64-int integer array per worker (Listing \ref{lst:single_scan2}).

The OpenCL kernels \lstinline{single_scan3} (Listing \ref{lst:single_scan3}), \lstinline{single_scan4} (Listing \ref{lst:single_scan4}), \lstinline{single_scan5} (Listing \ref{lst:single_scan5}), \lstinline{single_scan5_unroll} (Listing \ref{lst:single_scan5_unroll}), \lstinline{single_scan6} (Listing \ref{lst:single_scan6}) explore the GPU architecture to improve the sum of the partial distances. Each workgroup calculates the distance of several bitstrings. During the distance calculation, each worker calculates the exclusive OR (XOR) between two 64-bit integers and use the built-in popcount function to count the number of ones. Then, they update an array of partial distances with their results. This array is stored in the local memory and is shared between all workers of the same workgroup. This whole step happens simultaneously in the GPU. Then, a reduction algorithm is used to sum the partial distances array in order to calculate the total distance. This reduction algorithm is also distributed between the workers and runs in $O(\log_2(\text{bs\_step}))$. Finally, the first worker of each workgroup checks whether the distance is less than or equal to the radius to include the bitstring index into the resulting array.

Some of the optimizations may not work in some GPUs, because not all their premises are valid. Before choosing a kernel, one should check whether it works property for one's specific GPU device.


\begin{figure}[!p]
\begin{lstlisting}[
language={[OpenCL]C},
label={lst:single_scan0},
caption={OpenCL kernel {\lstinline[columns=fixed]{single_scan0}}. It calculates one distance per worker and let the GPU decide how to distribute this task between workgroups and workers. It is the most simple kernel and does not explore any details of the GPU architecture.},
]
__kernel
void single_scan0(
		__constant const uchar *bitcount_table,
		__global const ulong *bitstrings,
		const uint bs_len,
		const uint sample,
		__constant const ulong *bs,
		const uint radius,
		__global uint *counter,
		__global uint *selected,
		__local uint *partial_dist)
{
	uint id = get_global_id(0);

	if (id < sample) {
		ulong a;
		uint dist;

		const __global ulong *row = bitstrings + id*bs_len;

		dist = 0;
		for(uint j=0; j<bs_len; j++) {
			a = row[j] ^ bs[j];
			dist += popcount(a);
		}
		if (dist <= radius) {
			selected[atomic_inc(counter)] = id;
		}
	}
}
\end{lstlisting}
\end{figure}


\begin{figure}[!p]
\begin{lstlisting}[
language={[OpenCL]C},
label={lst:single_scan1},
caption={OpenCL kernel {\lstinline[columns=fixed]{single_scan1}}. It is just like \lstinline{single_scan0}, but it distribute several distances per workgroup, which, in turn, distribute the distances among their workers.},
]
__kernel
void single_scan1(
        __constant const uchar *bitcount_table,
        __global const ulong *bitstrings,
        const uint bs_len,
        const uint sample,
        __constant const ulong *bs,
        const uint radius,
        __global uint *counter,
        __global uint *selected,
        __local uint *partial_dist)
{
    uint id;
    ulong a;
    uint dist;
    const __global ulong *row;

    for (id=get_global_id(0); id < sample; id += get_global_size(0)) {

        row = bitstrings + id*bs_len;

        dist = 0;
        for(uint j=0; j<bs_len; j++) {
            a = row[j] ^ bs[j];
            dist += popcount(a);
        }
        if (dist <= radius) {
            selected[atomic_inc(counter)] = id;
        }

    }
}
\end{lstlisting}
\end{figure}


\begin{figure}[!p]
\begin{lstlisting}[
language={[OpenCL]C},
label={lst:single_scan2},
caption={OpenCL kernel {\lstinline[columns=fixed]{single_scan2}}. It calculates one distance per workgroup, distributing each 64-bit integer operation per worker, and then summing the results obtained by the workers. The sum algorith is done by only the first worker of each workgroup.},
]
__kernel
void single_scan2(
        __constant const uchar *bitcount_table,
        __global const ulong *bitstrings,
        const uint bs_len,
        const uint sample,
        __constant const ulong *bs,
        const uint radius,
        __global uint *counter,
        __global uint *selected,
        __local uint *partial_dist)
{
    uint dist;
    ulong a;
    uint j;

    for (uint id = get_group_id(0); id < sample; id += get_num_groups(0)) {

        const __global ulong *row = bitstrings + id*bs_len;

        dist = 0;
        j = get_local_id(0);
        if (j < bs_len) {
            a = row[j] ^ bs[j];
            dist += popcount(a);
        }
        partial_dist[get_local_id(0)] = dist;

        barrier(CLK_LOCAL_MEM_FENCE);

        if (get_local_id(0) == 0) {
            dist = 0;
            for(uint t = 0; t < bs_len; t++) {
                dist += partial_dist[t];
            }
            if (dist <= radius) {
                selected[atomic_inc(counter)] = id;
            }
        }

        barrier(CLK_LOCAL_MEM_FENCE);
    }
}
\end{lstlisting}
\end{figure}

\begin{figure}[!p]
\begin{lstlisting}[
language={[OpenCL]C},
label={lst:single_scan3},
caption={OpenCL kernel {\lstinline[columns=fixed]{single_scan3}}. It calculates one distance per workgroup, distributing each 64-bit integer operation per worker, and then summing the results obtained by the workers. The sum algorith is a parallel reduction, in which the workers split the array in two parts and sum the second part in the first part every loop. So, the sum is calculated in {$O(\log_2(\text{number of workers per workgroup}))$}. This kernel only works when the number of workers per workgroup is a power-of-2.},
]
__kernel
void single_scan3(
        __constant const uchar *bitcount_table,
        __global const ulong *bitstrings,
        const uint bs_len,
        const uint sample,
        __constant const ulong *bs,
        const uint radius,
        __global uint *counter,
        __global uint *selected,
        __local uint *partial_dist)
{
    uint dist;
    ulong a;
    uint j;

    for (uint id = get_group_id(0); id < sample; id += get_num_groups(0)) {

        const __global ulong *row = bitstrings + id*bs_len;

        dist = 0;
        j = get_local_id(0);
        if (j < bs_len) {
            a = row[j] ^ bs[j];
            dist = popcount(a);
        }
        partial_dist[get_local_id(0)] = dist;

        // Parallel reduction to sum all partial_dist array.
        for(uint stride = get_local_size(0)/2; stride > 0; stride /= 2) {
            barrier(CLK_LOCAL_MEM_FENCE);
            if (get_local_id(0) < stride) {
                partial_dist[get_local_id(0)] +=
                    partial_dist[get_local_id(0) + stride];
            }
        }

        if (get_local_id(0) == 0) {
            if (partial_dist[0] <= radius) {
                selected[atomic_inc(counter)] = id;
            }
        }

        barrier(CLK_LOCAL_MEM_FENCE);
    }
}
\end{lstlisting}
\end{figure}

\begin{figure}[!p]
\begin{lstlisting}[
language={[OpenCL]C},
label={lst:single_scan4},
caption={OpenCL kernel {\lstinline[columns=fixed]{single_scan4}}. This kernel is just like \lstinline{single_scan3}, but it works with any number of workers per workgroup. The tradeoff is that it includes an aditional step in the parallel reduction algorithm.},
]
__kernel
void single_scan4(
        __constant const uchar *bitcount_table,
        __global const ulong *bitstrings,
        const uint bs_len,
        const uint sample,
        __constant const ulong *bs,
        const uint radius,
        __global uint *counter,
        __global uint *selected,
        __local uint *partial_dist)
{
    uint dist;
    ulong a;
    uint j;

    for (uint id = get_group_id(0); id < sample; id += get_num_groups(0)) {

        const __global ulong *row = bitstrings + id*bs_len;

        dist = 0;
        j = get_local_id(0);
        if (j < bs_len) {
            a = row[j] ^ bs[j];
            dist = popcount(a);
        }
        partial_dist[get_local_id(0)] = dist;

        uint old_stride = get_local_size(0);
        __local uint extra;
        extra = 0;
        for(uint stride = get_local_size(0)/2; stride > 0; stride /= 2) {
            barrier(CLK_LOCAL_MEM_FENCE);
            if ((old_stride&1) == 1 && get_local_id(0) == old_stride-1) {
                extra += partial_dist[get_local_id(0)];
            }
            if (get_local_id(0) < stride) {
                partial_dist[get_local_id(0)] +=
                    partial_dist[get_local_id(0) + stride];
            }
            old_stride = stride;
        }

        if (get_local_id(0) == 0) {
            if (partial_dist[0] + extra <= radius) {
                selected[atomic_inc(counter)] = id;
            }
        }

        barrier(CLK_LOCAL_MEM_FENCE);
    }
}
\end{lstlisting}
\end{figure}

\begin{figure}[!p]
\begin{lstlisting}[
language={[OpenCL]C},
label={lst:single_scan5},
caption={OpenCL kernel {\lstinline[columns=fixed]{single_scan5}}. This kernel is just like \lstinline{single_scan3}, but it explores one more detail of many GPU microarchitecture: the size of the warp. As the workers in the same warp are always synchronized, there is no need to synchronize them using a barrier. This specific kernel only works when the number of workers per workgroup is a power-of-2.},
]
__kernel
void single_scan5(
        __constant const uchar *bitcount_table,
        __global const ulong *bitstrings,
        const uint bs_len,
        const uint sample,
        __constant const ulong *bs,
        const uint radius,
        __global uint *counter,
        __global uint *selected,
        __local uint *partial_dist)
{
    uint dist;
    ulong a;
    uint j;

    for (uint id = get_group_id(0); id < sample; id += get_num_groups(0)) {
        const __global ulong *row = bitstrings + id*bs_len;

        dist = 0;
        j = get_local_id(0);
        if (j < bs_len) {
            a = row[j] ^ bs[j];
            dist = popcount(a);
        }
        partial_dist[get_local_id(0)] = dist;

        uint stride;
        for(stride = get_local_size(0)/2; stride > 32; stride /= 2) {
            barrier(CLK_LOCAL_MEM_FENCE);
            if (get_local_id(0) < stride) {
                partial_dist[get_local_id(0)] +=
                    partial_dist[get_local_id(0) + stride];
            }
        }
        barrier(CLK_LOCAL_MEM_FENCE);
        for(/**/; stride > 0; stride /= 2) {
            if (get_local_id(0) < stride) {
                partial_dist[get_local_id(0)] +=
                    partial_dist[get_local_id(0) + stride];
            }
        }

        if (get_local_id(0) == 0) {
            if (partial_dist[0] <= radius) {
                selected[atomic_inc(counter)] = id;
            }
        }
        barrier(CLK_LOCAL_MEM_FENCE);
    }
}
\end{lstlisting}
\end{figure}

\begin{figure}[!p]
\begin{lstlisting}[
language={[OpenCL]C},
basicstyle=\scriptsize\ttfamily,
label={lst:single_scan5_unroll},
caption={OpenCL kernel {\lstinline[columns=fixed]{single_scan5_unroll}}. This kernel is exactly like \lstinline{single_scan5}, but it unrolls the last for since it has at most 5 loops.},
]
__kernel
void single_scan5_unroll(
        __constant const uchar *bitcount_table,
        __global const ulong *bitstrings,
        const uint bs_len,
        const uint sample,
        __constant const ulong *bs,
        const uint radius,
        __global uint *counter,
        __global uint *selected,
        __local uint *partial_dist)
{
    uint dist;
    ulong a;
    uint j;

    for (uint id = get_group_id(0); id < sample; id += get_num_groups(0)) {
        const __global ulong *row = bitstrings + id*bs_len;

        dist = 0;
        j = get_local_id(0);
        if (j < bs_len) {
            a = row[j] ^ bs[j];
            dist = popcount(a);
        }
        partial_dist[get_local_id(0)] = dist;

        for(uint stride = get_local_size(0)/2; stride > 32; stride /= 2) {
            barrier(CLK_LOCAL_MEM_FENCE);
            if (get_local_id(0) < stride) {
                partial_dist[get_local_id(0)] += partial_dist[get_local_id(0) + stride];
            }
        }

        // We do not need to sync because they all run in the same warp.
        if (get_local_id(0) < 32 && get_local_size(0) >= 64) {
            partial_dist[get_local_id(0)] += partial_dist[get_local_id(0) + 32];
        }
        if (get_local_id(0) < 16 && get_local_size(0) >= 32) {
            partial_dist[get_local_id(0)] += partial_dist[get_local_id(0) + 16];
        }
        if (get_local_id(0) < 8 && get_local_size(0) >= 16) {
            partial_dist[get_local_id(0)] += partial_dist[get_local_id(0) + 8];
        }
        if (get_local_id(0) < 4 && get_local_size(0) >= 8) {
            partial_dist[get_local_id(0)] += partial_dist[get_local_id(0) + 4];
        }
        if (get_local_id(0) < 2 && get_local_size(0) >= 4) {
            partial_dist[get_local_id(0)] += partial_dist[get_local_id(0) + 2];
        }

        if (get_local_id(0) == 0) {
            partial_dist[0] += partial_dist[1];
            if (partial_dist[0] <= radius) {
                selected[atomic_inc(counter)] = id;
            }
        }
        barrier(CLK_LOCAL_MEM_FENCE);
    }
}
\end{lstlisting}
\end{figure}

\begin{figure}[!p]
\begin{lstlisting}[
language={[OpenCL]C},
basicstyle=\scriptsize\ttfamily,
label={lst:single_scan6},
caption={OpenCL kernel {\lstinline[columns=fixed]{single_scan6}}. This kernel is just like \lstinline{single_scan5}, but it works with any number of workers per work. The tradeoff is an aditional step in the parallel reduction algorithm.},
]
__kernel
void single_scan6(
        __constant const uchar *bitcount_table,
        __global const ulong *bitstrings,
        const uint bs_len,
        const uint sample,
        __constant const ulong *bs,
        const uint radius,
        __global uint *counter,
        __global uint *selected,
        __local uint *partial_dist)
{
    uint dist;
    ulong a;
    uint j;

    for (uint id = get_group_id(0); id < sample; id += get_num_groups(0)) {
        const __global ulong *row = bitstrings + id*bs_len;

        dist = 0;
        j = get_local_id(0);
        if (j < bs_len) {
            a = row[j] ^ bs[j];
            dist = popcount(a);
        }
        partial_dist[get_local_id(0)] = dist;

        uint old_stride = get_local_size(0);
        uint stride;
        __local uint extra;
        extra = 0;
        for(stride = get_local_size(0)/2; stride > 32; stride /= 2) {
            barrier(CLK_LOCAL_MEM_FENCE);
            if ((old_stride&1) == 1 && get_local_id(0) == old_stride-1) {
                extra += partial_dist[get_local_id(0)];
            }
            if (get_local_id(0) < stride) {
                partial_dist[get_local_id(0)] +=
                    partial_dist[get_local_id(0) + stride];
            }
            old_stride = stride;
        }
        barrier(CLK_LOCAL_MEM_FENCE);
        for(/**/; stride > 0; stride /= 2) {
            if ((old_stride&1) == 1 && get_local_id(0) == old_stride-1) {
                extra += partial_dist[get_local_id(0)];
            }
            if (get_local_id(0) < stride) {
                partial_dist[get_local_id(0)] +=
                    partial_dist[get_local_id(0) + stride];
            }
            old_stride = stride;
        }

        if (get_local_id(0) == 0) {
            if (partial_dist[0] + extra <= radius) {
                selected[atomic_inc(counter)] = id;
            }
        }
        barrier(CLK_LOCAL_MEM_FENCE);
    }
}
\end{lstlisting}
\end{figure}

\section{Counters}

Each hard-location has one integer of data per bit. For instance, each hard-location of a 1,000 bits SDM has 1,000 bits. Those integers are stored in a counter.

A counter is an array of integers which stores the data of all hard locations. So, the counter's array has $n \cdot H$ integers.

When two counters are added in a third counter, there may occur an overflow. It is not supposed to be a problem because, by default, each counter is a signed 32-bit integer that can store any number between -2,147,483,648 and 2,147,483,647, which means they will not overflow with less writes than $2^{31}-1$ divided by the average number of activated hard locations. For instance, when $n=1,000$, $H=1,000,000$, and $r=451$, the average number of activated hard locations is 1,000 and it would require at least one million writes before being possible to a counter to overflow.  Note also that it would be more likely to saturate the memory before any overflow.

Anyway, counters may have overflow protection depending on compiling options. By default, there is no overflow check for performance reasons (and because it does not seem necessary).

\section{Read and write operations}

The reading and writing operations are executed in two steps: first, the address space is swept looking for the activated addresses; then, the operation is performed in the counters. Reading operation assemblies the bitstring according to the counters of the activated addresses, while the writing operation changes the counters.

The iterated reading keeps reading until it gets exactly the same bitstring (or the number of maximum interations has been reached), i.e., it performs $\eta_{i+1} = \text{read}(\eta_i)$ and stops when $\eta_{k+1} = \eta_{k}$. If the initial bitstring is inside the critical distance of $\eta$, it will converge to $\eta$, but, if it is not, it will diverge and reach the maximum number of iterations.

The framework has both Kanerva's read and Murilo's generalized read. The generalization brings a parameter $z$, which is the exponent. In this case, the results are floating point instead of integer, which considerably reduces performance. When $z=1$, it is exactly as the Kanerva's read. When $z=0$, it is the Chada's read. We also explored how SDM would behave for different values of $z$.

There is another special read operation: the weighted reading. In the weighted reading, the value of the counters are multiplied by a weight which depends only on the distance between the reading address and the hard-location address. The weight is retrieved from a lookup table of integers indexed by the distance. The rest of the read operation is exactly the same.

There is also a weighted writing operation. In this case, the weight is applied when the counters are updated, i.e., if the weight is 2, the counters are increased twice when bits are 1, and decreased twice when bits are 0. Just as in the weighted reading, the weights depend only on the distance between the writing address and the hard-location address. The weights are retrieved from a lookup table of integers indexed by the distance.



\chapter{Results (i): Framework validation}
% !TEX root = ../partial-sdm.tex

The framework has been validated comparing its results with the expected results from \citet{Kanerva1988}. Thus, we run simulations which were then compared to the theoretical analysis conducted some decades ago.

\section{Distance between random bitstrings}

As showed by \citet{Kanerva1988}, the distance between two bitstrings follows a binomial distribution with mean $\mu = n/2$ and standard deviation $\sigma = \sqrt{n}/2$. For large values of $n$, it may be approximated by a normal distribution with the same mean and standard deviation.

In order to validate our random bitstring generation algorithm, we have calculated 10,000 distances between two random bitstrings with $n=1,000$ bits. In total, 20,000 random bitstrings have been generated during the simulation. The code is available in the ``Distance between bitstrings'' notebook \citep{sdmframework}.

In figure \ref{fig:validation-distance}, we can notice that the theoretical model and the simulation matches. Hence, it seems the random bitstring generation algorithm works properly.

This also validates the algorithm used to calculate the distance between two bitstrings. In this simulation, we have used the built-in \_\_popcnt() function.

\begin{figure}[!htb]
  \centering
  \subfloat[Full histogram ]{\includegraphics[width=0.5\textwidth]{./images02/new-images/bs-hist-full.png}}
  \subfloat[{Zoom in the interval $[400, 600]$} ]{\includegraphics[width=0.5\textwidth]{./images02/new-images/bs-hist-zoom.png}}

  \caption{Histogram of 10,000 distances between two random bitstrings with 1,000 bits. The curve in red is the theoretical normal distribution with $\mu = 500$ and $\sigma = \sqrt{500}/2$.}
  \label{fig:validation-distance}
\end{figure}


\section{Number of activated hard-locations}

In his seminal work, Kanerva proposed to use a sample of 1,000,000 hard-locations in a 1,000 bits SDM. He also proposed to activate only 1,000 of them, on average. He calculated that an access radius of $r=451$ would activate, on average, 0.00107185004892 of the whole space, or, in this case, 1,071.85 hard-locations.

We extended his results, calculating the distribution of the number of activated hard-locations. As each hard-location has probability $p=0.00107185004892$ of being activated, the probability of activating exactly $a$ out of $H$ hard-locations follows a binomial distribution with mean $\mu = pH$ and standard deviation $\sigma = \sqrt{Hp(p-1)}$. In this case, $\mu = 1071.85$ and $\sigma = 32.72$.

In order to validate our scan algorithm, we have run 10,000 scans from a random bitstring and counted the number of activated hard-locations. The code is available in the ``Number of activated hard-locations'' notebook \citep{sdmframework}.

In figure \ref{fig:validation-activated-hls}, we can notice that the theoretical model and the simulation matches. Hence, it seems that both the address space generation algotihm and the scan algorithm work properly. Notice that the curve is almost the same for $n=1,000$ and $n=256$. It happens because the access radius is adjust to have $p$ as close as possible to $0.001$. They are not exactly the same because their $p$ differs a little.

\begin{figure}[!htb]
  \centering
  \subfloat[$n=1,000$, $H=1,000,000$,\protect\\ $r=451$, and $p=0.00107185$ ]{\includegraphics[width=0.5\textwidth]{./images02/new-images/activated-hls-1000.png}}
  \subfloat[$n=256$, $H=1,000,000$,\protect\\ $r=103$, and $p=0.00106684$ ]{\includegraphics[width=0.5\textwidth]{./images02/new-images/activated-hls-256.png}}

  \caption{Histogram of the number of activated hard-locations in 10,000 scans from a random bitstring. The curve in red is the theoretical normal distribution with $\mu = Hp$ and $\sigma = p(p-1)H$.}
  \label{fig:validation-activated-hls}
\end{figure}

Besides the number of activated hard-locations, we have also extended Kanerva's results to calculate the distribution of distances between the center of the circle and the activated hard-locations. Let $A$ be the set of activated hard-locations, $\xi$ be the center of the circle, and $r$ be the access radius, then:

\begin{align}
P(\text{d}(a, \xi)=x | a \in A) &= \frac{P(\text{d}(a, \xi)=x)}{P(a \in A)} \\
    &= \frac{\binom{n}{x}}{\sum_{k=0}^r \binom{n}{k}} \label{eq:prob-d-inside-circle}
\end{align}

In order to check Equation \ref{eq:prob-d-inside-circle}, we have calculated the distances of the activated hard-locations to the center of 1,000 random circles. The code is available in the ``Distances of activated hard-locations'' notebook \citep{sdmframework}.

In figure \ref{fig:validation-distance-activated-hls}, we can notice that the theoretical model and the simulation matches.

\begin{figure}[!htb]
\centering\includegraphics[width=\textwidth]{./images02/new-images/distance-activated-hls.png}

\caption{Histogram of the distances of activated hard-locations to the center of the circles. The curve in red is the theoretical distribution of Equation \ref{eq:prob-d-inside-circle}
\label{fig:validation-distance-activated-hls}}
\end{figure}

\section{Intersection of two circles}

Kanerva has calculated the intersection of two circles according to the distance between their centers. The intersection is important to understand how SDM works, because it affects directly the critical distance. When $\eta_d$ is inside the critical distance, then it will converge to $\eta$. In fact, it converges because they share a sufficient amount of hard-locations, i.e., the intersection of the circle around $\eta_d$ and $\eta$ is enough to converge. For further information about the relation between the critical distance and the intersection, see \citet{brogliato2014sparse}.

We have calculated the intersection between a random bitstring (bs1) and another bitstrings (bs2) exactly $d$ bits away. The former (bs1) is just a random bitstring. The latter (bs2) was generated randomly flipping $d$ bits of bs1. The code is available in the ``Kanerva's Figure 1.2'' notebook \citep{sdmframework}.

In Figure \ref{fig:validation-intersection}, we can notice that we have obtained the same results as Kanerva. It seems that the random flipping bits algorithm and the scan algorithm work properly.

\begin{figure}[!htb]
  \centering
  \subfloat[{\citet[Figure 1.2, p.25]{Kanerva1988}} ]{\includegraphics[width=0.45\textwidth]{./images02/new-images/kanerva-table-12.jpg}}
  \subfloat[Generated by SDM-Frameworkm with $n=1,000$ ]{\includegraphics[width=0.55\textwidth]{./images02/new-images/intersection-of-circles.png}}

  \caption{Number of hard-locations in the intersection of circles around two bitstrings $x$ bits away.}
  \label{fig:validation-intersection}
\end{figure}


\section{Storage and retrieval of sequences}

\citet[Ch.8]{Kanerva1988} presented an approach to store and retrieve sequences using $k$ different SDMs, namely $\text{sdm}_1$, $\text{sdm}_2$, $dots$, $\text{sdm}_k$.

Let $a_0, a_1, a_2, \dots, a_n$ be a sequence to be stored in a $k$-fold memory. So, all pointers of the form $a_i \rightarrow a_{i+k}$ will be written to $\text{sdm}_k$ memory, i.e., in $\text{sdm}_1$, the following pointers will be written: $a_0 \rightarrow a_1$, $a_1 \rightarrow a_2$, $\dots$, $a_{n-1} \rightarrow a_n$; while in $\text{sdm}_2$, the following pointers will be written: $a_0 \rightarrow a_2$, $a_2 \rightarrow a_3$, $\dots$, $a_{n-2} \rightarrow a_n$; and so forth.

We have tested exactly the same example presented in \citet{Kanerva1988}, p.85. We wrote two sequences to a $3$-fold memory: $<A, B, C, D>$ and $<E, B, C, F>$. Then, after reading the sequences $<A, B, C>$ and $<E, B, C>$, we have obtained $D$ and $F$, respectively.

Each reading operation was performed summing the counters of all activated hard-locations from all three memories. For instance, to read the sequence $<A, B, C>$, we have activated the hard-locations around $C$ in $\text{sdm}_1$, we have also activated the hard-locations around $D$ in $\text{sdm}_2$, and, finally, we have also activated the hard-locations around $A$ in $\text{sdm}_3$. After summing the counters of all those hard-locations, we evaluate the resulting bitstring just as in the original read operation.

The code is available in the ``Sequences (Kanerva Ch 8)'' notebook \citep{sdmframework}.

The logic behind how it works is that, when reading the sequence $<A, B, C>$, we have $A$ pointing to $D$, while both $B$ and $C$ point to $D$ and $F$. Thus, $D$ appears more often than $F$ and ended up being the result.

Hence, as we have replicated the theoretical results from Kanerva, we have one more evidence that our framework works properly.

\subsection{$k$-fold memory using only one SDM}

We have extended Kanerva's ideas to be able to store and retrieve sequences in $k$-fold memories using only one SDM (instead of $k$ SDMs).

Our idea was to create $k$ random bitstrings, one for each fold. We have performed writing and reading exactly as Kanerva's original idea, but, instead of writing to $\text{sdm}_k$, we have written $a_{i+k}$ into the address $a_i \oplus \text{tag}_k$, and, instead of reading from $\text{sdm}_k$, we have read from address $a_i \oplus \text{tag}_k$, where $\oplus$ is the exclusive or (XOR) operator.

It worked as if we had splitted SDM into $k$ regions with low intersection between two of them. So, as the interference is minimal, they work like independent SDMs. The major disadvantage of this approach is that memory capacity may be reached faster.

Splitting the memory into regions may be an interesting strategy to other sorts of problems, mostly the ones which would need many SDMs and, consequently, would use a lot of RAM.



\chapter{Results (ii): Critical distance}
% !TEX root = ../partial-sdm.tex

One particular analysis of Kanerva's interest is given by the limits of recovery.  That is, given an item read at a distance $x$ from a previously stored $\eta$, does this reading at a $\eta_x$ recover the original? Suppose an SDM is trying to read an item written at $\eta$, but the cues received so far lead to a point of distance $x$ from $\eta$.  As one reads at $\eta_x$, a new bitstring $\beta$ is obtained, leading to Kanerva's question: what is the new distance from $\eta$ to $\beta$? Is it smaller or larger than $x$? That, of course, depends on the ratio between $x$ and the number of dimensions of the memory.

\citet[p.70]{Kanerva1988} originally predicted a \textasciitilde 500-bit distance after a point (Figure \ref{fig:kanerva-figure-7.3}). The original prediction considered that the read distance would decline when inside the critical distance and increase afterward, converging to a \textasciitilde 500-bit distance.  At this point, each read would lead to a different, orthogonal, \textasciitilde 500-bit distance bitstring. He analyzed specifically an SDM with 1,000 bits and 10,000 random bitstrings written into it.

\begin{figure}[h]
\centering\includegraphics[width=0.8\textwidth]{images02/kanerva-table-7-2-original.png}
\caption{Kanerva's original Figure 7.3 (p. 70) predicting a \textasciitilde 500-bit distance after a point.
\label{fig:kanerva-figure-7.3}}
\end{figure}

As we ran the simulations, this one, in particular, struck our attention: The new distances obtained after a read operation were not perfectly predicted by the theoretical model. We have strictly followed Kanerva's configuration and, even so, we have found out some deviations from Kanerva's original theoretical analysis and the results obtained by simulation.

In details, we have created a SDM with $n=1,000$, $H=1,000,000$, and $r=451$. Then, we have generated 10,000 random bitstrings and written them into the memory. Then, we have generated a reference bitstring (bs\_ref) and written it into the memory. Then, we have executed the following steps with $x$ from 0 to 1,000: (i) copy bs\_ref into a new bitstring; (ii) randomly flipped $x$ bits of the copy; (iii) read from the memory in the copy address; and (iv) stored the distance between the returned bitstring and bs\_ref. Finally, we have plotted Figure \ref{fig:sdm-10000w-table-7-2}.

\begin{figure}[h]
\centering
\subfloat[1 sample for each distance $x$ \label{fig:sdm-10000w-table-7-2-1sample} ]{\includegraphics[width=0.5\textwidth]{./images02/sdm-10000w-table-7-2.png}}
\subfloat[6 samples for each distance $x$ \label{fig:sdm-10000w-table-7-2-6samples} ]{\includegraphics[width=0.5\textwidth]{./images02/sdm-10000w-table-7-2-6-samples.png}}

\caption{Results generated by the framework diverging from Kanerva's original Figure 7.3. Here we had a 1,000 bit, 1,000,000 hard location SDM with 10,000 random bitstrings written into it, which was also Kanerva's configuration.
\label{fig:sdm-10000w-table-7-2}}
\end{figure}

Figure \ref{fig:sdm-10000w-table-7-2-1sample} has a lot of noise because we have read only once for each distance $x$ and Kanerva has predicted the average distance. So, we have changed the steps to run $k$ reads and store the average new distance. We run with $k=6$, and the results can be seen in Figure \ref{fig:sdm-10000w-table-7-2-6samples}, which has much lower noise and still holds the divergence.

Our results show that the theoretical prediction is not accurate.  There are interaction effects from one or more of the attractors created by the 10,000 writes, and these attractors seem to raise the distance beyond \textasciitilde 500 bits (Figure \ref{fig:sdm-10000w-table-7-2}).

Obviously, these small deviations from Kanerva's original theoretical predictions deserve a qualification.  Kanerva was working in the 1980s and the 1990s, and had no access to the immense computational power that we do today. It is no surprise that some small interaction effects should exist as machines allow us to explore the ideas of his monumental work.

However, when we reduced the number of random bitstrings written in the SDM from 10,000 to only 100, the results reflected very well the Kanerva's theoretical expectation (Figure \ref{fig:sdm-100w-table-7-2-10samples}). This result strengthens our hypothesis that the disparities in the computational outcomes are due to the interaction effect of high numbers of different attractors. In Figure \ref{fig:sdm-table-7-2-steps} we can notice that the more random bitstrings are written, the stronger the attractors.

\begin{figure}[h]
\centering
\subfloat[100 writes \label{fig:sdm-100w-table-7-2-10samples} ]{\includegraphics[width=0.47\textwidth]{./images02/sdm-100w-table-7-2.png}}
\subfloat[Steps of 1,000 writes \label{fig:sdm-table-7-2-steps} ]{\includegraphics[width=0.53\textwidth]{./images02/sdm-table-7-2-steps.png}}
%\centering\includegraphics[width=\textwidth]{images02/sdm-100w-table-7-2.png}

\caption{Results generated by the framework similar to Kanerva's original Figure 7.3. Here we have a 1,000 bit, 1,000,000 hard location SDM with (a) just 100 random bitstrings written into it and (b) steps of 1,000 random bitstrings written into it.
\label{fig:sdm-100w-table-7-2}}
\end{figure}

To obtain the results from Figures \ref{fig:sdm-10000w-table-7-2} and \ref{fig:sdm-100w-table-7-2}, we had to write 10,000 random bitstrings to an SDM, and then randomly choose one of those bitstrings to be our origin. Finally, we randomly flipped some bits from the origin bitstring and executed a reading operation in the SDM. Thereby, in order to show the interaction effects more clearly, we changed the single read for a 15-iterative read. As we can see in Figure \ref{fig:sdm-10000w-table-7-2-15iter}, after a distance of 500 bits, all bitstrings converged to 500-bit distance bitstrings, just as described by Kanerva.

Hence, our understanding is that the attractors are just preventing the bitstrings to converge directly to 500-bit distance bitstrings, requiring more reading steps to do so. They are in other orthogonal bitstrings' critical distance, but sufficiently far not to converge in a single read.

\begin{figure}[h]
\centering\includegraphics[width=\textwidth]{images02/sdm-10000w-table-7-2-15iter.png}
\caption{This graph shows the interaction effects more clearly.  As we change the single read to a 6-iterative read, the effect has vanished, and all bitstrings above $x=500$ have converged to 500-bit distance bitstrings. Here we have precisely the same configuration of Figure \ref{fig:sdm-10000w-table-7-2}, except for the iterative read.
\label{fig:sdm-10000w-table-7-2-15iter}}
\end{figure}

Going further in the analysis, we calculated the probability of missing a bit when reading from SDM. After all, that is how Kanerva has originally found the curve. To do this, we used the following equations from our previous work \citep{brogliato2014sparse}. Let $d$ be the distance to the target, $h$ be the number of hard locations activated during reading and write operations, $s$ be the number of total stored bitstrings, $H$ be the number of total hard locations, $w$ be the number of times the target was written into SDM, $\theta$ be the total random bitstrings in all $h$ hard locations activated by read operation, and $\phi(d)$ be the average number of shared hard locations activated two bitstrings $d$ bits away.

\begin{align}
\theta &= \frac{sh^2}{H} - w \cdot \phi(d) \\
P(miss | bit=0) &= 1 - P \left( \sum_{i=1}^\theta X_i < \frac{sh^2}{2H} \right) \\
P(miss | bit=1) &= P \left( \sum_{i=1}^\theta X_i < \frac{sh^2}{2H} - w \cdot \phi(d) \right) \\
P(miss) &= \frac{1}{2} \cdot \left[ P(miss | bit=0) + P(miss | bit=1) \right]
\end{align}

For details and the proof of this equation, see \citet{brogliato2014sparse}. Although Kanerva has found a formula for $\phi(d)$ through an unsolved integral, and \citet{de1995geometrical} have proposed another way to calculate $\phi(d)$, we have used our framework to estimate the values of $d$. In order to do that, we used a Monte Carlo approach, generating many pairs of random bitstrings $d$ bits away from them and calculating the average number of shared hard locations between them. The code is available in the ``Calculate critical distance'' notebook \citep{sdmframework}.

Kanerva's settings according to the parameters of the equation were: $s=10,000$, $H=1,000,000$, and $w=1$. We have calculated $\phi(d)$ as explained, and $h = H \cdot 2^{-n} \sum_{i=0}^{r} \binom{n}{i}$, where $n=1,000$ and $r=451$. Finally, $h=1,071.85$ and changing $d$ from $0$ to $1000$, we got Figure \ref{fig:kanerva-figure-73-calculated}.

\begin{figure}[!htb]
\centering\includegraphics[width=0.8\textwidth]{./images02/calculated-table-72.png}
\caption{Kanerva's original Figure 7.3 generated using the equations from \citet{brogliato2014sparse}.
\label{fig:kanerva-figure-73-calculated}
}
\end{figure}

As one can easily notice, we have got exactly the same curve as Kanerva. Both his and our model expect that, after reading, say, from 550 bits of distance from a written bitstring, we should obtain the expected $n/2$ equator distance. This question has intrigued us, and here we look for a more analytic explanation than merely interference from the other written attractors. Let us turn back to mathematics to study this anomaly.






% !TEX root = ../partial-sdm.tex

\section{A deviation from the equator distance?}

Kanerva writes\footnote{Email thread `SDM: A puzzling issue and an invitation', started March 16th 2018, in which we discussed the aforementioned discrepancy.  To think that some centuries ago, all scientific publishing was the exchange of such letters.}:

\begin{quote}
    You have done an incredibly thorough analysis of SDM. I like the puzzle in your message and believe that your simulations are correct and to be learned from.  So what to make of the difference compared to my Figure 7.3 (and your Figure \ref{fig:kanerva-figure-73-calculated})?  I think the difference comes from my not having accounted fully for the effect of the other 9,999 vectors that are stored in the memory.  You say in it\\

   ``Our results show that the theoretical prediction is not accurate. There are interaction effects from one or more of the attractors created by the 10,000 writes, and these attractors seem to raise the distance beyond 500 bits (Figure \ref{fig:sdm-10000w-table-7-2}).'' \\

   I think that is correct.  It also brings to mind a comment Louis Jaeckel made when we worked at NASA Ames.  He pointed out that autoassociative storage (each vector is stored with itself as the address) introduces autocorrelation that my formula for Figure 7.2 did not take into account.  When we read from memory, each stored vector exerts a pull toward itself, which also means that each bit of a retrieved vector is slightly biased toward the same bit of the read address, regardless of the read address.  We never worked out the math.
\end{quote}

This is an important observation. A hard location is activated because it shares many dimensions with the items read from or written onto it. Imagine the `counter's eye view':  each individual counter `likes'\footnote{If you are a cell I then I have some serious piece of advice to you.  I would strongly suggest avoiding the neighborhood of a neuron. Neurons have serious authoritarian tendencies, bossing around --- generally electrifying --- the hopeless innocent cells around them just because.} to write on its own corresponding bit-address value more than it likes the opposite; as each hard-location has a say in its own area --- and nowhere else.

Let $x$ and $y$ be random bitstrings and $n$ be the number of dimensions in the memory; let $x_i$ and $y_i$ be the $i$-th bit of $x$ and $y$, respectively; and $d(x, y)$ be the Hamming distance. Whilst the probability of a shared bit-value between same dimension-bits in two random addresses is $1/2$, an address only activates hard-locations close to it.  Let us call these shared bitvalues a \emph{bitmatch in dimension $i$}.

So, what is the probability of bitmatches given that we know the access radius $r$ between the address and a hard-location?
\bigskip

\begin{theorem}
\emph{Each dimension has a small pull bias, which can be measured by}
\label{T1}
$P(x_i = y_i | d(x, y) \le r) = \dfrac{\sum_{k=0}^{r} \binom{n-1}{k}}{\sum_{k=0}^{r} \binom{n}{k}}.$
\end{theorem}

\begin{proof}
    The left-hand expression $P(x_i = y_i | d(x, y) \le r)$ computes the probability of a bitmatch in $i$, given that we know that $x$ and $y$ are in the access radius defined by $r$, i.e., $d(x, y)\le r$.

    Applying the law of total probability to the left-hand expression we obtain

    \begin{align}
    \sum_{k=0}^{r} P(x_i = y_i | d(x, y) = k \le r) P(d(x, y) = k | d(x, y) \le r)
    \end{align}

    We also know that

    \begin{align}
    P(x_i = y_i | d(x, y) = k) &= \frac{n-k}{n} \\
    P(d(x, y) = k | d(x, y) \le r) &= \frac{\binom{n}{k}}{\sum_{j=0}^{r} \binom{n}{j}}
    \end{align}

    Hence,

    \begin{align}
    P(x_i = y_i | d(x, y) \le r) = \frac{\sum_{k=0}^{r} \frac{n-k}{n} \binom{n}{k}}{\sum_{j=0}^{r} \binom{n}{j}}
    \end{align}

    Finally, the combinatorial identity

    \begin{align}
    \frac{n-k}{n} \binom{n}{k} = \frac{(n-k)}{n} \frac{n!}{(n-k)! k!} = \frac{(n-1)!}{k! (n-1-k)!} = \binom{n-1}{k}
    \end{align}

    closes the theorem.

\end{proof}

Theorem \ref{T1} is valid for both ``x written at x'' (autoassociative memory) and ``random written at x'' (heteroassociative memory). When $n=1,000$ and $r=451$, $P(x_i = y_i | d(x, y) \le r) = p = 0.552905498137$.  Each bit of a hard location does indeed have a small pull bias.  What is meant by this is that each particular dimension has a small preference toward positive values if its address bit is set to 1, and negative values if set to 0.


\begin{lemma}
Let $r$ be the access radius given that $f$ percent of the hard locations are activated. Then, $\lim_{n \rightarrow \infty} r/n = 1/2$.
\label{thm:sdm-access-radius}
\end{lemma}
\begin{proof}

As the bits of the hard locations' addresses are randomly chosen, the distance between two hard locations follow a Binomial distribution with $n$ samples and probability 0.5 ($B(n, 0.5)$). For $n$ sufficiently large, the Binomial distribution can be approximated by a Normal distribution, i.e., $B(n, 0.5) \rightarrow \mathcal{N}(\mu = n/2, \sigma^2 = n/4)$.

Let $\Phi(x)$ be the cdf of the standard normal distribution. Let $z = \frac{r - n/2}{\sqrt{n}/2}$. Thus, $P(d(x, y) \le r) = \Phi(z)$. As $f = P(d(x, y) \le r)$, then, $f = \Phi(z)$.

Calculating the inverse, $z = \Phi^{-1}(f)$. Then,

\begin{align}
z &= \Phi^{-1}(f) \\
\frac{r - n/2}{\sqrt{n}/2} &= \Phi^{-1}(f) \\
r &= \frac{n}{2} + \Phi^{-1}(f) \frac{\sqrt{n}}{2} \\
\frac{r}{n} &= \frac{1}{2} + \Phi^{-1}(f) \frac{1}{2 \sqrt{n}}
\end{align}

Therefore, $n \rightarrow \infty \Rightarrow r/n \rightarrow 1/2$.
\end{proof}

\begin{theorem}
The autocorrelation vanishes when $n \rightarrow \infty$, i.e., $\lim_{n \rightarrow \infty} P(x_i = y_i | d(x, y) \le r) = 1/2$.
\label{thm:sdm-autocorrelation-convergence}
\end{theorem}
\begin{proof}

From Lemma \ref{thm:sdm-access-radius}, we know that $r = n/2$ for $n$ sufficiently large.

Suppose that $n$ is an even integer, then,

$$
\sum_{k=0}^{r} \binom{n}{k} = \sum_{k=0}^{n/2} \binom{n}{k} = \frac{1}{2} \sum_{k=0}^{n} \binom{n}{k} = \frac{2^n}{2} = 2^{n-1}
$$

And, also,

\begin{align*}
\sum_{k=0}^{r} \binom{n-1}{k} &= \sum_{k=0}^{n/2} \binom{n-1}{k} \\
    &= \frac{1}{2} \left[ \sum_{k=0}^{n-1} \binom{n-1}{k} - \binom{n-1}{n/2} \right] \\
    &= \frac{1}{2} \left[ 2^{n-1} - \binom{n-1}{n/2} \right] \\
    &= 2^{n-2} - \frac{1}{2} \binom{n-1}{n/2}
\end{align*}

Finally,

\begin{align}
P(x_i = y_i | d(x, y) \le r) &= \frac{\sum_{k=0}^{r} \binom{n-1}{k}}{\sum_{k=0}^{r} \binom{n}{k}} \\
    &= \frac{2^{n-2} - \frac{1}{2} \binom{n-1}{n/2}}{2^{n-1}} \\
    &= \frac{2^{n-2}}{2^{n-1}} - \frac{1}{2^n} \binom{n-1}{n/2} \\
    &= \frac{1}{2} - \frac{1}{2^n} \binom{n-1}{n/2}
\end{align}

As $n \rightarrow \infty \Rightarrow \frac{1}{2^n} \binom{n-1}{n/2} \rightarrow 0$, the proof is done for $n$ even.

When $n$ is an odd integer, the steps of the proof are similar. Therefore, the proof is complete.

\end{proof}

In Figure \ref{fig:sdm-10kbits-figure-7-3}, $n=10,000$ and we can notice that the autocorrelation has vanished as predicted by Theorem \ref{thm:sdm-autocorrelation-convergence}.

\begin{figure}[!htb]
  \centering
  \includegraphics[width=0.75\textwidth]{./images02/sdm-10000bits-10000w-table-7-2.png}

  \caption{Given an address $x$ and a dimension $i$, how many hard locations with bitmatches in $i$ are activated by reading at $x$?  The histogram was obtained through numerical simulation. The red curve is the theoretical normal distribution found in Theorem \ref{T2}.}
  \label{fig:sdm-10kbits-figure-7-3}
\end{figure}

So far we have looked only at a single pair of bitstrings, the probability of a single bitmatch between bitstrings within the access radius distance.  Now let us consider the number of activated hard locations exhibiting this bitmatch.

Let $h$ be the number of activated hard locations. As the probability of activating a specific hard location is constant, $h \sim \text{Binomial}(H, p_1)$. Thus, $\mathbf{E}[h] = \mu_h = Hp_1$ and $\mathbf{V}[h] = \sigma^2_h = Hp_1(1-p_1)$, where $p_1 = 2^{-n} \sum_{k=0}^{r} \binom{n}{k}$.

Let $Z$ be the number of activated hard locations with the same bit as the reading address. Then, $Z = \sum_{i=1}^{h} X_i$, where $X_i \sim \text{Bernoulli}(p)$, where $p = P(x_i = y_i | d(x, y) \le r)$.

\begin{theorem}
Given a reading address $x$ and a dimension $i$, the number of activated hard-locations with bitmatches at $i$ follows a normal distribution with $\mathbf{E}[Z] = \mu_Z = p \mu_h$ and $\mathbf{V}[Z] = \sigma_Z^2 = p(1-p) \mu_h + p^2 \sigma^2_h$.
\label{T2}
\end{theorem}

\begin{proof}
As $P(973 < h < 1170) = 0.997$, by the central limit theorem, $Z$ may be approximated by a normal distribution.

By the central limit theorem, $Z$ is normally distributed.

Applying the law of total averages and the law of total variance, $\mathbf{E}[Z] = \mathbf{E}[\mathbf{E}[Z | h]] = \mathbf{E}[ph] = p \mathbf{E}[h] = ph$, and $\mathbf{V}[Z] = \mathbf{E}[\mathbf{V}[Z|h]] + \mathbf{V}[\mathbf{E}[Z|h]] = \mathbf{E}[hp(1-p)] + \mathbf{V}[ph] = p(1-p) \mathbf{E}[h] + p^2 \mathbf{V}[h] = hp(1-p) + p^2 H p_1 (1-p_1)$.

Applying the law of total variance, $\mathbf{V}[Z] = \mathbf{E}[\mathbf{V}[Z|h]] + \mathbf{V}[\mathbf{E}[Z|h]] = \mathbf{E}[hp(1-p)] + \mathbf{V}[ph] = p(1-p) \mathbf{E}[h] + p^2 \mathbf{V}[h] = p(1-p)\mu_h + p^2 \sigma^2_h$.
\end{proof}


See Figure \ref{fig:sdm-same-bit-histogram} for a comparison between the theoretical model and a simulation.

\begin{figure}[h!]
  \centering
  \includegraphics[width=0.75\textwidth]{./images02/autocorrelation/same-bit-histogram.png}

  \caption{Given an address $x$ and a dimension $i$, how many hard locations with bitmatches in $i$ are activated by reading at $x$?  The histogram was obtained through numerical simulation. The red curve is the theoretical normal distribution found in Theorem \ref{T2}.}
  \label{fig:sdm-same-bit-histogram}
\end{figure}

\section{Counter bias}

The previous theorems show that there is bias in the counters. Let us analyze the $i$th counter of a hard location.

Let $s$ be the number of bitstrings written into memory (in our case, $s=10,000$) and $\text{addr}_i$ be the $i$th bit of the hard location's address.

Let $\theta$ be the average number of bitstrings written in each hard location. As there are $s$ bitstrings written into the memory, and the probability of activating a specific hard location is constant, $\theta \sim \text{Binomial}(s, p_1)$. Thus, $\mathbf{E}[\theta] = \mu_\theta = s p_1$ and $\mathbf{V}[\theta] = \sigma^2_\theta = s p_1 (1 - p_1)$.

Let $Y_i$ be the number of bitmatches in the $i$ of a hard location's address after $s$ written bitstrings. Then, $Y_i = \sum_{k=1}^{\theta} X_k$.

\begin{theorem}
Given the number of written bitstrings $s$, $\mathbf{E}[Y_i] = \mu_Y = p \mu_\theta$ and $\mathbf{V}[Y_i] = \sigma^2_Y = p(1-p) \mu_\theta + p^2 \sigma^2_\theta$.
\end{theorem}
\begin{proof}
Applying the law of total expectation, $\mathbf{E}[Y] = \mathbf{E}[\mathbf{E}[Y|\theta]] = \mathbf{E}[p \theta] = p \mathbf{E}[\theta] = p \mu_\theta$.

Applying the law of total variance, $\mathbf{V}[Y] = \mathbf{E}[\mathbf{V}[Y|\theta]] + \mathbf{V}[\mathbf{E}[Y|\theta]] = \mathbf{E}[\theta p (1-p)] + \mathbf{V}[p \theta] = p(1-p) \mathbf{E}[\theta] + p^2 \mathbf{V}[\theta] = p(1-p) \mu_\theta + p^2 \sigma^2_\theta$.
\end{proof}

During a write operation, the counters are incremented for every bit 1 and decremented for every bit 0. So, after $s$ writes, there will be $\theta$ bitstrings written in each hard location with $Y_i$ bitmatches and $\theta - Y_i$ non-bitmatches. Thus, $[\text{cnt}_i | \text{addr}_i = 1] = (Y_i) - (\theta - Y_i) = 2Y_i - \theta$ and $[\text{cnt}_i | \text{addr}_i = 0] = \theta - 2Y_i$.

\begin{theorem}
$\mathbf{E}[\text{cnt}_i | \text{addr}_i = 1] = \mu_{\text{cnt}} = (2p-1) \mu_\theta$ and $\mathbf{V}[\text{cnt}_i | \text{addr}_i = 1] = \sigma^2_\text{cnt} = 4p(1-p) \mu_\theta + (2p-1)^2 \sigma^2_\theta$.
\end{theorem}

\begin{proof}

$\mathbf{E}[\text{cnt}_i | \text{addr}_i = 1] = \mathbf{E}[2Y_i - \theta] = \mathbf{E}[2Y_i] - \mathbf{E}[\theta] = 2 \mathbf{E}[Y_i] - \mu_\theta = 2 p \mu_\theta - \mu_\theta = (2p-1) \mu_\theta$.

Applying the law of total variance, $\mathbf{V}[\text{cnt}_i | \text{addr}_i = 1] = \mathbf{V}[2Y_i - \theta] = \mathbf{E}[\mathbf{V}[2Y_i - \theta | \theta]] + \mathbf{V}[\mathbf{E}[2Y_i - \theta | \theta]]$.

Let us solve each part independently. Thus,

$\mathbf{V}[2Y_i - \theta | \theta] = \mathbf{V}[2Y_i | \theta] = 4 \mathbf{V}[Y_i | \theta] = 4 \mathbf{V}[\sum_{k=1}^\theta X_k] = 4 \theta p (1-p)$.

$\mathbf{E}[\mathbf{V}[2Y_i - \theta | \theta]] = \mathbf{E}[4 \theta p (1-p)] = 4p(1-p) \mathbf{E}[\theta] = 4p(1-p) \mu_\theta$.

Finally,

$\mathbf{E}[2Y_i - \theta | \theta] = 2 \mathbf{E}[Y_i | \theta] - \mathbf{E}[\theta | \theta] = 2p \theta - \theta = (2p-1) \theta$.

$\mathbf{V}[\mathbf{E}[2Y_i - \theta | \theta]] = \mathbf{V}[(2p-1) \theta] = (2p-1)^2 \mathbf{V}[\theta] = (2p-1)^2 \sigma^2_\theta$.
\end{proof}

\begin{theorem}
$\mathbf{E}[\text{cnt}_i | \text{addr}_i = 0] = - \mu_{\text{cnt}}$ and $\mathbf{V}[\text{cnt}_i | \text{addr}_i = 1] = \sigma^2_\text{cnt}$.
\end{theorem}
\begin{proof}
Notice that $[\text{cnt}_i | \text{addr}_i = 0] = -[\text{cnt}_i | \text{addr}_i = 1]$. Thus, $\mathbf{E}[\text{cnt}_i | \text{addr}_i = 0] = -\mathbf{E}[\text{cnt}_i | \text{addr}_i = 1]$ and $\mathbf{V}[\text{cnt}_i | \text{addr}_i = 0] = \mathbf{V}[\text{cnt}_i | \text{addr}_i = 1]$.
\end{proof}

In summary,

\begin{align}
\left[ \text{cnt}_i | \text{addr}_i=1 \right] &\sim \mathcal{N}(\mu_\text{cnt}, \sigma^2_\text{cnt})\label{cntaddr1} \\
\left[ \text{cnt}_i | \text{addr}_i=0 \right] &\sim \mathcal{N}(-\mu_\text{cnt}, \sigma^2_\text{cnt})\label{cntaddr0}
\end{align}

In our case, $p=0.5529$, $s=10,000$, and $H=1,000,000$, so $[\text{cnt}_i | \text{addr}_i=1] \sim \mathcal{N}(\mu=1.1341, \sigma^2 = 10.7184)$. For ``random at x'', $p=0.5$, so $\mu = 0$ and $\sigma^2 = 10.7185$. See Figure \ref{fig:sdm-corr-counters}.

\begin{figure}[h!]
  \centering
  \subfloat[$\text{addr}_i=1$]{\includegraphics[width=0.5\textwidth]{./images02/autocorrelation/x_at_x_addr1.png}}
  \subfloat[$\text{addr}_i=0$]{\includegraphics[width=0.5\textwidth]{./images02/autocorrelation/x_at_x_addr0.png}}

  \caption{The value of the counters after $s=10,000$ writes shows the autocorrelation in the counters in autoassociative memories (``x at x''). The histogram was obtained through simulation. The red curve is the theoretical normal distribution found in equations (\ref{cntaddr1}) and (\ref{cntaddr0}).}
  \label{fig:sdm-corr-counters}
\end{figure}


Finally,

\begin{align}
P(\text{cnt}_i > 0 | \text{addr}_i = 1) = P(\text{cnt}_i < 0 | \text{addr}_i = 0) = 1 - \mathcal{N}.\text{cdf}(0)
\end{align}

For ``random written at x'', $p=0.5$ implies $\mu_\text{cnt} = 0$, which implies $P(\text{cnt}_i > 0 | \text{addr}_i = 1) = P(\text{cnt}_i < 0 | \text{addr}_i = 0) = 0.5$, independently of the parameters because they will only affect the variance and the normal distribution is symmetrical around the average.

However, for ``x written at x'', $p=0.5529$ and the probabilities depend on $s$. For $s=10,000$, they are equal to 0.6354. For $s=20,000$, they are equal to 0.6867. For $s=30,000$, they are equal to 0.7232. The more random bitstrings are written into the memory, the more the hard locations point to themselves.

Let $D$ be the number of counters aligned with $\text{addr}_i$. The standard deviation was calculated using the fact that $[D|\theta] \sim \text{Binomial}(1000, q=P(\text{cnt}_i > 0 | \text{addr}_i=1, \theta))$.

Applying the law of total variance, $\mathbf{V}[D] = \mathbf{E}[\mathbf{V}[D|\theta]] + \mathbf{V}[\mathbf{E}[D|\theta]] = \mathbf{E}[1000 q (1-q)] + \mathbf{V}[1000 q] = 1000 \mathbf{E}[q-q^2] + 1000^2 \mathbf{V}[q] = 1000 \mathbf{E}[q](1-\mathbf{E}[q]) + 1000(1000-1)\mathbf{V}[q]$, where $\mathbf{E}[q] = \sum_\theta P(\text{cnt}_i > 0 | \text{addr}_i=1, \theta) P(\theta)$ and $\mathbf{E}[q^2] = \sum_\theta [P(\text{cnt}_i > 0 | \text{addr}_i=1, \theta)]^2 P(\theta)$.

Doing the math, $\mathbf{E}[q] = 0.402922$ and $\mathbf{E}[q^2] = 0.634433$. Thus, $\mathbf{V}[q] = \mathbf{E}[q^2] - (\mathbf{E}[q])^2 = 0.0004166$. Hence, $\mathbf{V}[D] = 648.2041$. See Figure \ref{fig:sdm-corr-prob} and notice that I still have to figure out why the mean is correct, but the standard deviation is not.

\begin{figure}[h!]
  \centering
  \subfloat[``random at x'']{\includegraphics[width=0.5\textwidth]{./images02/autocorrelation/random_at_x_counters.png}}
  \subfloat[``x at x'']{\includegraphics[width=0.5\textwidth]{./images02/autocorrelation/x_at_x_counters.png}}

  \caption{Autocorrelation in the counters in autoassociative memories (``x written at x''). The histogram was obtained through simulation. The red curve is the theoretical distribution.}
  \label{fig:sdm-corr-prob}
\end{figure}



\section{Read bias}

Now that we know the distribution of $\text{cnt}_i | \text{addr}_i$, we may go to the read operation. During the read operation, on average, $h$ hard locations are activated and their counters are summed up. So, for the $i$th bit,

\begin{align}
\text{acc}_i = \sum_{k=1}^{h} \text{cnt}_k
\end{align}

Let $\eta$ be the reading address and $\eta_i$ the $i$th bit of it. Then, let's split the $h$ activated hard locations into two groups: (i) the ones with the same bit as $\eta_i$ with $Z$ hard locations, and (ii) the ones with the opposite bit as $\eta_i$ with $h-Z$ hard locations.

\begin{align}
\left[ \text{acc}_i|\eta_i \right] &= \sum_{k=1}^{Z} \left[ \text{cnt}_k | \text{addr}_k=\eta_i \right] + \sum_{k=1}^{h-Z} \left[ \text{cnt}_k | \text{addr}_k \ne \eta_i \right]
\end{align}

Each sum is a sum of normally distributed random variables, so

\begin{align}
\sum_{k=1}^{Z} \left[ \text{cnt}_k | \text{addr}_k=\eta_1 \right] &\sim \mathcal{N}(\mu = \mu_\text{cnt} \mu_Z, \sigma^2 = \sigma_\text{cnt}^2 \mu_Z + \mu_\text{cnt}^2 \sigma^2_Z) \label{eqn:sdm-eta1-addr1} \\
\sum_{k=1}^{h-Z} \left[ \text{cnt}_k | \text{addr}_k \ne \eta_1 \right] &\sim \mathcal{N}(\mu = -\mu_\text{cnt} (1-p) \mu_h, \sigma^2 = \sigma^2_\text{cnt} (1-p) \mu_h + \mu_\text{cnt}^2 \sigma^2_{h-Z}) \label{eqn:sdm-eta1-addr0}
\end{align}

In our case, $\sum_{k=1}^{Z} \left[ \text{cnt}_k | \text{addr}_k=1 \right] \sim \mathcal{N}(\mu=672.12, \sigma^2=7113.87)$, and $\sum_{k=1}^{Z} \left[ \text{cnt}_k | \text{addr}_k=1 \right] \sim \mathcal{N}(\mu=-543.49, \sigma^2=5752.54)$. See Figure \ref{fig:sdm-read-sums} --- we can notice that the average is correct but the variance is too small.

\begin{figure}[h!]
  \centering
  \subfloat[Equation \ref{eqn:sdm-eta1-addr1}  ($\text{addr}_k=1)$]{\includegraphics[width=0.5\textwidth]{./images02/autocorrelation/read-counters-eta1_addr1.png}}
  \subfloat[Equation \ref{eqn:sdm-eta1-addr0} ($\text{addr}_k=0$)]{\includegraphics[width=0.5\textwidth]{./images02/autocorrelation/read-counters-eta1_addr0.png}}

  \caption{The histogram was obtained through simulation. The red curve is the theoretical normal distribution.}
  \label{fig:sdm-read-sums}
\end{figure}

Hence,

\begin{align}
\left[ \text{acc}_i|\eta_i=1 \right] &\sim \mathcal{N}(\mu = (2p-1)^2 \mu_\theta \mu_h, \sigma^2 = \sigma_\text{cnt}^2 \mu_h + 2 \mu_\text{cnt}^2 \sigma^2_h) \label{eqn:sdm-eta1} \\
\left[ \text{acc}_i|\eta_i=0 \right] &\sim \mathcal{N}(\mu = -(2p-1)^2 \mu_\theta \mu_h, \sigma^2 = \sigma_\text{cnt}^2 \mu_h + 2 \mu_\text{cnt}^2 \sigma^2_h) \label{eqn:sdm-eta0}
\end{align}

In our case, $\left[ \text{acc}_i|\eta_i=1 \right] \sim \mathcal{N}(\mu = 128.62, \sigma^2 = 12865.69)$, and $\left[ \text{acc}_i|\eta_i=0 \right] \sim \mathcal{N}(\mu = -128.62, \sigma^2 = 12865.69)$. See Figure \ref{fig:sdm-read} --- we can notice that the variance issue from Figure \ref{fig:sdm-read-sums} has propagated to these images.

\begin{figure}[h!]
  \centering
  \subfloat[Equation \ref{eqn:sdm-eta1}  ($\eta_k=1)$]{\includegraphics[width=0.5\textwidth]{./images02/autocorrelation/read-counters-eta1.png}}
  \subfloat[Equation \ref{eqn:sdm-eta0} ($\eta_k=0$)]{\includegraphics[width=0.5\textwidth]{./images02/autocorrelation/read-counters-eta0.png}}

  \caption{The histogram was obtained through simulation. The red curve is the theoretical normal distribution.}
  \label{fig:sdm-read}
\end{figure}


Finally,

\begin{align}
P(wrong) &= P(\text{acc}_i < 0 | \eta_i = 1) \cdot P(\eta_i = 1) + P(\text{acc}_i > 0 | \eta_i = 0) \cdot P(\eta_i = 0) \\
    &= \frac{\mathcal{N}_{\eta_i=1}.\text{cdf}(0)}{2} + \frac{1-\mathcal{N}_{\eta_i=0}.\text{cdf}(0)}{2} \\
    &= \frac{\mathcal{N}_{\eta_i=1}.\text{cdf}(0)}{2} + \frac{\mathcal{N}_{\eta_i=1}.\text{cdf}(0)}{2} \\
    &= \mathcal{N}_{\eta_i=1}.\text{cdf}(0)
\end{align}

Using the empirical variance of $\sigma^2 = 27838.3029124$, we calculate $P(wrong) = 0.22037771219874325$.

In order to check this probability, I have run a simulation reading from 1,000 random bitstrings (which have never been written into memory) and calculate the distance from the result of a single read. As the $P(wrong) = 0.22037$, I expected to get an average distance of 220.37 with a standard deviation of 13.10. See Figure \ref{fig:sdm-read-random-bs} for the comparison between the simulated and the theoretical outcomes.

Figure \ref{fig:sdm-single-read-dist} shows the new distance between $\eta_d$ and $\text{read}(\eta_d)$, where $\eta_d$ is $d$ bits away from $\eta$. As for $d \ge 520$ there is no intersection between $\eta$ and $\eta_d$, our models applies and explains the horizontal line around distance 220.

\begin{figure}[!htb]
  \centering
  \includegraphics[width=0.6\textwidth]{./images02/autocorrelation/read-random-bs.png}

  \caption{The histogram was obtained through simulation. The red curve is the theoretical normal distribution.}
  \label{fig:sdm-read-random-bs}
\end{figure}

\begin{figure}[!htb]
  \centering
  \includegraphics[width=0.6\textwidth]{./images02/autocorrelation/single-read-dist.png}

  \caption{New distance after a single read operation in a bitstring $\eta_d$, which is $d$ bits away from $\eta$. The new distance was calculated between $\eta_d$ and $\text{read}(\eta_d)$. Notice that when $d \ge 520$, the intersection between $\eta$ and $\eta_d$ is zero, which means there is only random bitstrings written into the activated hard locations. The distance 220 equals $1000 \cdot 0.220$ which is the probability find in Figure \ref{fig:sdm-read-random-bs}.}
  \label{fig:sdm-single-read-dist}
\end{figure}


















\section{Critical distance of 209}

The critical distance is defined as $d$ where $P(miss) = d/n$, or, in Figure \ref{fig:kanerva-figure-73-calculated}, the point where the curve meets with the identity function (the black diagonal line). Thus, we plot a zoom-in of Figure \ref{fig:kanerva-figure-73-calculated} around $d=209$ in Figure \ref{fig:figure-73-eq-zoom-in} using the same equations \citep{brogliato2014sparse}. It was surprising that the meeting does not happen at $d=209$, but around $d=221$.

\begin{figure}[!htb]
\centering\includegraphics[width=0.8\textwidth]{./images02/figure-73-eq-zoom.png}
\caption{Zoom-in around $d=209$ of Figure \ref{fig:kanerva-figure-73-calculated}.
\label{fig:figure-73-eq-zoom-in}
}
\end{figure}

To confirm that the critical distance is not around 209, but around 221, we also plot a zoom-in of Figure \ref{fig:sdm-10000w-table-7-2} around $d=209$ in Figure \ref{fig:sdm-10000w-zoom}. In order to reduce the noise, we increased the samples to $k=180$.

\begin{figure}[!htb]
\centering\includegraphics[width=0.8\textwidth]{./images02/sdm-10000w-zoom-209.png}
\caption{Zoom-in around $d=209$ of Figure \ref{fig:sdm-10000w-table-7-2}.
\label{fig:sdm-10000w-zoom}
}
\end{figure}


%To obtain the results from Figures \ref{fig:sdm-10000w-table-7-2} and \ref{fig:sdm-100w-table-7-2}, we had to write 10,000 random bitstrings to an SDM, and then randomly choose one of those bitstrings to be our origin. Finally, we randomly flipped some bits from the origin bitstring and executed a reading operation in the SDM. Thereby, in order to show the interaction effects more clearly, we wrote a handmade bitstring to the SDM which had all bits inverted in relation to the origin bitstring --- their hamming distance was equal to 1,000. Our handmade bitstring was acting as an opposite attractor, and one can see the accelerating effects towards convergence to both attractors: the origin and the handmade bitstrings (Fig. \ref{sdm-10000w-notX-table-7-2}). Here we had the exact same configuration of Figure \ref{sdm-10000w-table-7-2}, with the addition of the single opposite attractor.

%\begin{figure}[h]
%\centering\includegraphics[width=0.8\textwidth]{images02/sdm-10000w-notX-table-7-2.png}
%\caption{This graph shows the interaction effects more clearly.  As we include an opposite bitstring, one can see the accelerating effects towards convergence to both attractors: the origin and its polar opposite. Here we have the exact same configuration of Figure \ref{sdm-10000w-table-7-2}, with the addition of the single opposite attractor.
%\label{sdm-10000w-notX-table-7-2}}
%\end{figure}



\chapter{Results (iii): Loss of neurons}
In SDM, the data is written distributed among millions of hard-locations, which theoretically gives SDM robustness against loss of neurons. In other words, SDM should keep convering correctly even when some neurons are dead. The question is: how robust it really is? How many neurons may die before it starts to forget things. These questions have never been addresses before.

Looking for answers to these questions, we run simulations in which we kept killing some neurons and checking whether SDM remained converging to a given bitstring or not. In these simulations, 10,000 random bitstrings were written to a 1,000-bit SDM with 1,000,000 hard-locations, and we choose one of them as our target. As the bitstrings were all written exactly once, we may generalize the results. The code is available in the ``Reseting hard-locations'' notebook \citep{sdmframework}.

As neurons are hard-locations in SDM, when we say that a neuron has been killed, we mean that its counters have been zeroed and a new random bitstring address has been assigned. During our simulations, no other bitstring has been written after the 10,000. Consequently, as their counters willl remain zeroed, it is exactly like ignoring the dead hard-locations in the subsequent reading operations.

\begin{figure}[!p]
\centering\includegraphics[width=\textwidth]{./images02/new-images/sdm-neuron-death-200k.png}
\caption{This graph shows the SDM's robustness against loss of neurons in a SDM with $n=1,000$ and $H=1,000,000$. It shows that a loss of 200,000 neurons, 20\% of the total, does not affect SDM whatsoever.
\label{fig:sdm-neuron-death-200k}}
\end{figure}

In Figure \ref{fig:sdm-neuron-death-200k}, we can notice that SDM is absolutely robust up to 200,000 neuron deaths which is 20\% of all hard-locations. This result is pretty impressive and really surprised us.

In fact, SDM begins to be significantly affected by loss of neurons after 600,000 neuron deaths (Figure \ref{fig:sdm-neuron-death-1m}), and obviously forgets everything when all neurons are dead.

\begin{figure}[!p]
\centering\includegraphics[width=\textwidth]{./images02/new-images/sdm-neuron-death-1m.png}
\caption{This graph shows the SDM's robustness against loss of neurons in a SDM with $n=1,000$ and $H=1,000,000$. The more neurons are death, the smaller the critical distance, i.e., the worse the SDM recall.
\label{fig:sdm-neuron-death-1m}}
\end{figure}

It is interesting that 500,000 neuron deaths has a minor effect in SDM's recall capability (see Figure \ref{fig:sdm-neuron-death-500k}). It is analogous to do an hemispherectomy in a person and, after the procedure, the person is able to recall and learn almost just like before. In fact, there are clinical reports of children submitted to hemispherectomy who live an almost normal life with minor motor problems.

\begin{figure}[!p]
\centering\includegraphics[width=\textwidth]{./images02/new-images/sdm-neuron-death-500k.png}
\caption{This graph shows the SDM's robustness against loss of neurons in a SDM with $n=1,000$ and $H=1,000,000$. Even when 50\% of neurons are dead, SDM recall is barely affected, which is an impressive result and matches with some clinical results of children submitted to hemispherectomy.
\label{fig:sdm-neuron-death-500k}}
\end{figure}

An important observation is that around 800,000 neuron deaths (80\% of all neurons) the critical distance is really small, i.e., SDM recall capacity is very diminished. After 900,000 neuron deaths the critical distance is zero. In this case, everything has been forgot.

Although there is some decrease in SDM recall after 600,000 neuron deaths, it is curious that there is a sudden change between 900,000 (90\%) and 1,000,000 (100\%). In Figure \ref{fig:sdm-neuron-death-details} we can see the details of this non-linear change. Notice that after 950,000 even the exact clue $\eta_0$ does not converge to $\eta$.

\begin{figure}[!p]
\centering\includegraphics[width=\textwidth]{images02/new-images/sdm-neuron-death.png}
\caption{This graph shows the SDM's robustness against loss of neurons in a SDM with $n=1,000$ and $H=1,000,000$.
\label{fig:sdm-neuron-death-details}}
\end{figure}

We run exactly the same simulation for a 256-bit SDM with 1,000,000 hard-locations. The results were even more surprising, because the 256-bit SDM was more robust to loss of neuron than the 1,000-bit SDM (see Figure \ref{fig:sdm-neuron-death-256bits}). Notice that the loss of 50\% of neurons barely affected the 256-bit SDM which remained functional even facing an enormous loss of 90\% of neurons.

\begin{figure}[!p]
\centering
\subfloat[Up to 500,000 neuron deaths ]{\includegraphics[width=\textwidth]{images02/new-images/sdm-neuron-death-256-500k.png}}

\subfloat[From 600,000 to 1,000,000 neuron deaths ]{\includegraphics[width=\textwidth]{images02/new-images/sdm-neuron-death-256-1m.png}}

\caption{This graph shows the SDM's robustness against loss of neurons in a SDM with $n=256$ and $H=1,000,000$.
\label{fig:sdm-neuron-death-256bits}}
\end{figure}


\chapter{Results (iv): Generalized read operation}

Murilo observed that the models of Kanerva's read ($z=1$) and Chada's read ($z=0$) were simple variations of a generalized read with an exponent $z$, which suggests experimenting with different values. Mathematically, let $A$ be the set of the counters of the activated hardlocation, and $c_i$ be the counter of the $i$-th bit. Then,

$$
s_i = \sum_{c \in A} \frac{c_i}{|c_i|} |c_i|^z
$$

The sum of $|c_i|^z$ turns the intermediate values from integers to float point numbers. Thus, we have developed a specific read operation which stored the intermediate values in double variables.

The results, however, have not yielded performance improvements. Though for $z \leq 1$ results are comparable to $z=1$, for $z>1$, the system shows a clear deterioration, with a smaller critical distance and faster divergence at large-distance reads. This is shown in Figures \ref{fig:murillo-generalization-experiments} and \ref{fig:murillo-generalization-experiments-6reads}.

We understand that the critical distant is an important parameter of SDM. The bigger the critical distance, the best, because SDM is able to converge even with farther clues. For $z>1$, the bigger the $z$, the smaller the critical distance. For $z = 6$, the critical distance almost reaches zero.

It is interesting that Kanerva has proposed $z=1$ without realizing the generalized reading. Even so, he proposed the optimal $z$ --- the one with the highest critical distance.

\begin{figure}[h!]
  \centering
  \subfloat[SDM behavior when $z \in \{0.1, 0.2, 0.3, 0.4, 0.5, 1\}$ ]{\includegraphics[width=\textwidth]{./images02/new-images/iter_z_01-05.png}}

  \subfloat[SDM behavior when $z \in \{1.5, 3, 4.5, 6\}$ ]{\includegraphics[width=\textwidth]{./images02/new-images/z_15_3_45_6.png}}

  \caption{(a) and (b) show the behavior of a single read. As stated previously, we can see a deterioration of convergence, with lower critical distance as $z>1$.  Another observation can be made here, concerning the discrepancy of Kanerva's Fig 7.3 and our data.  It seems that Kanerva may not have considered that a single read would only `clean' a small number of dimensions \emph{after the critical distance}. What we observe clearly is that with a single read, as the distance grows, the system only `cleans' towards the orthoghonal distance 500 after a number of iterative readings.}
  \label{fig:murillo-generalization-experiments}
\end{figure}

\begin{figure}[h!]
  \centering
  \subfloat[$z \in \{0, 1\}$]{\includegraphics[width=\textwidth]{./images02/new-images/z_0_1.png}}

  \subfloat[$z \in \{0, 0.5, 1, 1.5, 3, 4.5, 6\}$]{\includegraphics[width=\textwidth]{./images02/new-images/iter_z_all_2.png}}

  \caption{(a) and (b) show the behavior of Figure \ref{fig:murillo-generalization-experiments}, now executed with 6-iterative reads. What we observe clearly is that with a single read, as the distance grows, the system only `cleans' towards the orthoghonal distance 500 after a number of iterative readings.}
  \label{fig:murillo-generalization-experiments-6reads}
\end{figure}


\chapter{Results (v): Performance}
% !TEX root = ../partial-sdm.tex

Performance matters --- and has always mattered. If an experiment takes a few seconds, there is no point arguing whether we should try it. If it takes a few hours, maybe we should think first. If it takes a few days, it is more important to make a good plan. As SDM consumes a lot of both processor and memory resources, some experiments may take a long time.

Each scan a 1,000-bits SDM with 1,000,000 hard-locations executes $10^9$ bit compares through $10^9/64 = 15,625,000$ XORs and calls to the built-in popcount. So, 10,000 writes execute $10^{13}$ bit compares, while a 6-iterative reading executes $6 \cdot 10^{12}$ bit compares. The heatmap of Figure \ref{fig:cir-dist-10k-writes-kanerva} executed $3.05 \cdot 10^{15}$ bit compares. For comparison, the number of seconds since Jesus's birth is 63,639,648,000 = $6.36 \cdot 10^{10}$. The number of people who have ever lived on earth is estimated to be $1,08 \cdot 10^{11}$. There are approximtely $1.8 \cdot 10^{9}$ websites in the internet. A modern laptop can increment a counter $5 \cdot 10^{8}$ times per second. Hence, a naive implementation of SDM may take several hours to simply write 10,000 random bitstrings.

Amazon EC2 p3.2xlarge has generated the heatmap of Figure \ref{fig:cir-dist-10k-writes-kanerva} in 15 minutes and 3 seconds. It has compared $3.37 \cdot 10^{12}$ bits per second through $52.6 \cdot 10^{9} = 52.6 \text{ billion}$ XORs and popcounts per second. It is a great improvement when compared to the first versions of the code which took 16 hours to generate the same heatmap (and its memory use was already optimized and the computations were distributed in threads).

We have executed the same performance test in each device. The test has 3 parts. The first part consists of: 1,000 linear scans, 5,000 thread scans, and 5,000 OpenCL scans. The second part consists of 5,000 writes, 5,000 single reads, and 1,000 6-iterative readings, with both thread and opencl scanners. The code is available in the ``Performance test'' notebook \citep{sdmframework}.

Our first device is a personal MacBook Pro Retina 13-inch Late 2013 with a 2.6GHz Intel core i5 processor, 6GB DDR3 RAM, and Intel Iris GPU.

Our second device is an iMac Retina 5K 27-inch 2017 with a 3.8GHz Intel core i5 processor, 8GB DDR4 RAM, and a Radeon Pro 580 8G CPU. The complete results are presented in Figures \ref{fig:perf-imac-read-256}, \ref{fig:perf-imac-write-256}, \ref{fig:perf-imac-read-1000}, \ref{fig:perf-imac-write-1000}, \ref{fig:perf-imac-scan-256}, \ref{fig:perf-imac-scan-1000}, and \ref{fig:perf-imac-scan-10k} and Tables \ref{tab:perf-imac-256}, \ref{tab:perf-imac-1000}, and \ref{tab:perf-imac-10k}.

Beyond that, we are also running in state-of-the-art devices: (i) an Amazon EC2 p2.xlarge with Intel Xeon E5-2686v4 processor, 61GB DDR3 RAM, and NVIDIA K80 GPU, and (ii) an Amazon EC2 p3.2xlarge with Intel Xeon E5-2686v4 processor, 488GB DDR3 RAM, and NVIDIA Tesla V100 GPU.

It is interesting that which scanner is faster depends also on the SDM settings. In the iMac 2017, the faster scanner for a 1,000-bits SDM was the OpenCL, but for a 256-bit SDM was the threads. What happened here is that the OpenCL kernel chosen was a generic one which performs the scan for any SDM. It is always possible to optimize the OpenCL kernel to a specific setting, and it would be faster than the threads. By default, the framework chooses a generic kernel which we believe would be reasonable for the most common setups.
% TODO Run the optimized kernel and show the results.

Most of the time, the bottleneck of both the read and the write operations is the scanner. But we have optimized the OpenCL kernel so much for the iMac 2017 that scanner not the bottleneck anymore. In the writing operation, it took the same time to scan and to update the counters. It is impressive because it update, on average, 1,000 counters of 32-bit integers each, and it was as fast as (i) sending the command to the GPU, (ii) performing 1 billion bit compares, and (iii) downloading the response from the GPU.

Our conclusion is that, if one is really concerned about performance, one should fine tune the OpenCL kernel for one's GPU. It would always be faster than running in the CPU.


Kernel comparisons

\begin{table}[!htb]
\centering
\begin{tabular}{| l | r | r | r |}
    \hline
    & Loops & Scans / second & Scan time (ms) \\ \hline
    single\_scan0 & 3000 & 199.62 & 5.00 \\
    single\_scan1 & 3000 & 252.64 & 3.95 \\
    single\_scan2 & 3000 & 218.38 & 4.57 \\
    single\_scan3 & 3000 & 271.15 & 3.68 \\
    single\_scan4 & 3000 & 247.28 & 4.04 \\
    single\_scan5 & 3000 & 268.29 & 3.72 \\
    single\_scan5\_unroll & 3000 & 276.70 & 3.6139983336130777 \\
    single\_scan6 & 3000 & 248.26 & 4.02 \\
    \hline
\end{tabular}
\caption{iMac Retina 5K 27-inch 2017 with a 3.8GHz Intel core i5 processor, 8GB DDR4 RAM, and a Radeon Pro 580 8G GPU. Running an SDM with $n=1,000$ bits, $H=1,000,000$, and $r=451$.
\label{tab:perf-imac-256}}
\end{table}

\begin{table}[!htb]
\centering
\begin{tabular}{| l | r | r | r |}
    \hline
    & Loops & Scans / second & Scan time (ms) \\ \hline
    single\_scan0 & 3000 & 329.71 & 3.03 \\
    single\_scan1 & 3000 & 347.42 & 2.87 \\
    single\_scan2 & 3000 & 261.73 & 3.82 \\
    single\_scan3 & 3000 & 268.16 & 3.72 \\
    single\_scan4 & 3000 & 223.18 & 4.48 \\
    single\_scan5 & 3000 & 265.59 & 3.76 \\
    single\_scan5\_unroll & 3000 & 263.24 & 3.79 \\
    single\_scan6 & 3000 & 228.92 & 4.36 \\
\end{tabular}
\caption{iMac Retina 5K 27-inch 2017 with a 3.8GHz Intel core i5 processor, 8GB DDR4 RAM, and a Radeon Pro 580 8G GPU. Running an SDM with $n=256$ bits, $H=1,000,000$, and $r=103$.
\label{tab:perf-imac-256}}
\end{table}

\begin{table}[!htb]
\centering
\begin{tabular}{| l | r | r | r |}
    \hline
    & Loops & Scans / second & Scan time (ms) \\ \hline
    single\_scan0 & 500 & 16.37 & 61.06 \\
    single\_scan1 & 500 & 22.24 & 44.96 \\
    single\_scan2 & 500 & 22.23 & 44.98 \\
    single\_scan3 & 500 & 78.92 & 12.67 \\
    single\_scan4 & 500 & 87.30 & 11.45 \\
    single\_scan5 & 500 & 79.43 & 12.58 \\
    single\_scan5\_unroll & 500 & 87.88 & 11.37 \\
    single\_scan6 & 500 & 91.21 & 10.96 \\
\end{tabular}
\caption{iMac Retina 5K 27-inch 2017 with a 3.8GHz Intel core i5 processor, 8GB DDR4 RAM, and a Radeon Pro 580 8G GPU. Running an SDM with $n=10,000$ bits, $H=1,000,000$, and $r=4850$.
\label{tab:perf-imac-256}}
\end{table}

\begin{table}[!htb]
\centering
\begin{tabular}{| l | r | r | r |}
    \hline
    & Loops & Scans / second & Scan time (ms) \\ \hline
    single\_scan0 & 1000 & 42.36 & 23.60 \\
    single\_scan1 & 1000 & 75.63 & 13.22 \\
    single\_scan2 & 1000 & 21.14 & 47.28 \\
    single\_scan3 & 1000 & 30.24 & 33.06 \\
    single\_scan4 & 1000 & 24.79 & 40.32 \\
    single\_scan5 & 1000 & 31.73 & 31.51 \\
    single\_scan5\_unroll & 1000 & 36.28 & 27.55 \\
    single\_scan6 & 1000 & 25.32 & 39.48 \\
\end{tabular}
\caption{MacBook Pro Retina 13-inch Late 2013 with a 2.6GHz Intel core i5 processor, 6GB DDR3 RAM, and Intel Iris GPU. Running an SDM with $n=1,000$ bits, $H=1,000,000$, and $r=451$.
\label{tab:perf-imac-256}}
\end{table}

\begin{table}[!htb]
\centering
\begin{tabular}{| l | r | r | r |}
    \hline
    & Loops & Scans / second & Scan time (ms) \\ \hline
    single\_scan0 & 1000 & 119.49 & 8.36 \\
    single\_scan1 & 1000 & 95.82 & 10.43 \\
    single\_scan2 & 1000 & 42.58 & 23.48 \\
    single\_scan3 & 1000 & 39.19 & 25.51 \\
    single\_scan4 & 1000 & 23.58 & 42.39 \\
    single\_scan5 & 1000 & 40.94 & 24.42 \\
    single\_scan5\_unroll & 1000 & 43.91 & 22.77 \\
    single\_scan6 & 1000 & 23.70 & 42.18 \\
\end{tabular}
\caption{MacBook Pro Retina 13-inch Late 2013 with a 2.6GHz Intel core i5 processor, 6GB DDR3 RAM, and Intel Iris GPU. Running an SDM with $n=256$ bits, $H=1,000,000$, and $r=103$.
\label{tab:perf-imac-256}}
\end{table}

\begin{table}[!htb]
\centering
\begin{tabular}{| l | r | r | r |}
    \hline
    & Loops & Scans / second & Scan time (ms) \\ \hline
    Loops & Total time & Scans per second & Time per scan (ms) \\
    single\_scan0 & 3000 & 1433.27 & 0.69 \\
    single\_scan1 & 3000 & 1822.51 & 0.54 \\
    single\_scan2 & 3000 & 1165.06 & 0.85 \\
    single\_scan3 & 3000 & 979.93 & 1.02 \\
    single\_scan4 & 3000 & 906.35 & 1.10 \\
    single\_scan5 & 3000 & 1049.21 & 0.95 \\
    single\_scan6 & 3000 & 958.50 & 1.04 \\
\end{tabular}
\caption{EC2 p3.2xlarge 1000 bits
\label{tab:perf-imac-256}}
\end{table}

\begin{table}[!htb]
\centering
\begin{tabular}{| l | r | r | r |}
    \hline
    & Loops & Scans / second & Scan time (ms) \\ \hline
    Loops & Total time & Scans per second & Time per scan (ms) \\
    single\_scan0 & 3000 & 2729.58 & 0.36 \\
    single\_scan1 & 3000 & 2703.49 & 0.36 \\
    single\_scan2 & 3000 & 1354.12 & 0.73 \\
    single\_scan3 & 3000 & 1573.43 & 0.63 \\
    single\_scan4 & 3000 & 951.22 & 1.05 \\
    single\_scan5 & 3000 & 1598.69 & 0.62 \\
    single\_scan6 & 3000 & 986.06 & 1.01 \\
\end{tabular}
\caption{EC2 p3.2xlarge 256 bits
\label{tab:perf-imac-256}}
\end{table}

\begin{table}[!htb]
\centering
\begin{tabular}{| l | r | r | r |}
    \hline
    & Loops & Scans / second & Scan time (ms) \\ \hline
    Loops & Total time & Scans per second & Time per scan (ms) \\
    single\_scan0 & 500 & 48.54 & 20.60 \\
    single\_scan1 & 500 & 202.34 & 4.94 \\
    single\_scan2 & 500 & 199.29 & 5.01 \\
    single\_scan3 & 500 & 164.66 & 6.07 \\
    single\_scan4 & 500 & 166.70 & 5.99 \\
    single\_scan5 & 500 & 186.26 & 5.36 \\
    single\_scan6 & 500 & 167.74 & 5.96 \\
\end{tabular}
\caption{EC2 p3.2xlarge 10k bits
\label{tab:perf-imac-256}}
\end{table}

\begin{table}[!htb]
\centering
\begin{tabular}{| l | r | r | r |}
    \hline
    & Loops & Scans / second & Scan time (ms) \\ \hline
    Loops & Total time & Scans per second & Time per scan (ms) \\
    single\_scan0 & 3000 & 263.61 & 3.79 \\
    single\_scan1 & 3000 & 253.45 & 3.94 \\
    single\_scan2 & 3000 & 117.06 & 8.54 \\
    single\_scan3 & 3000 & 102.70 & 9.73 \\
    single\_scan4 & 3000 & 87.82 & 11.38 \\
    single\_scan5 & 3000 & 100.46 & 9.95 \\
    single\_scan6 & 3000 & 88.21 & 11.33 \\
\end{tabular}
\caption{EC2 p2.xlarge 1000 bits
\label{tab:perf-imac-256}}
\end{table}

\begin{table}[!htb]
\centering
\begin{tabular}{| l | r | r | r |}
    \hline
    & Loops & Scans / second & Scan time (ms) \\ \hline
    Loops & Total time & Scans per second & Time per scan (ms) \\
    single\_scan0 & 3000 & 1312.85 & 0.76 \\
    single\_scan1 & 3000 & 1237.20 & 0.80 \\
    single\_scan2 & 3000 & 178.85 & 5.59 \\
    single\_scan3 & 3000 & 158.43 & 6.31 \\
    single\_scan4 & 3000 & 97.14 & 10.29 \\
    single\_scan5 & 3000 & 149.45 & 6.69 \\
    single\_scan6 & 3000 & 97.10 & 10.29 \\
\end{tabular}
\caption{EC2 p2.xlarge 256 bits
\label{tab:perf-imac-256}}
\end{table}

\begin{table}[!htb]
\centering
\begin{tabular}{| l | r | r | r |}
    \hline
    & Loops & Scans / second & Scan time (ms) \\ \hline
    Loops & Total time & Scans per second & Time per scan (ms) \\
    single\_scan0 & 500 & 28.20 & 35.45 \\
    single\_scan1 & 500 & 17.30 & 57.80 \\
    single\_scan2 & 500 & 15.69 & 63.71 \\
    single\_scan3 & 500 & 25.04 & 39.92 \\
    single\_scan4 & 500 & 21.97 & 45.49 \\
    single\_scan5 & 500 & 22.97 & 43.51 \\
    single\_scan6 & 500 & 24.13 & 41.42 \\
\end{tabular}
\caption{EC2 p2.xlarge 10k bits
\label{tab:perf-imac-256}}
\end{table}

\begin{figure}[!htb]
\centering\includegraphics[width=\textwidth]{images02/performance/imac-read-256.png}
\caption{Time to run a single read in a SDM with $n=256$, $H=1,000,000$ and $r=103$.
\label{fig:perf-imac-read-256}}
\end{figure}

\begin{figure}[!htb]
\centering\includegraphics[width=\textwidth]{images02/performance/imac-write-256.png}
\caption{Time to run one write in a SDM with $n=256$, $H=1,000,000$ and $r=103$.
\label{fig:perf-imac-write-256}}
\end{figure}

\begin{figure}[!htb]
\centering\includegraphics[width=\textwidth]{images02/performance/imac-write-1000.png}
\caption{Time to run one write in a SDM with $n=1,000$, $H=1,000,000$ and $r=451$.
\label{fig:perf-imac-write-1000}}
\end{figure}

\begin{figure}[!htb]
\centering\includegraphics[width=\textwidth]{images02/performance/imac-read-1000.png}
\caption{Time to run a single read in a SDM with $n=1,000$, $H=1,000,000$ and $r=451$.
\label{fig:perf-imac-read-1000}}
\end{figure}

\begin{figure}[!htb]
\centering\includegraphics[width=\textwidth]{images02/performance/imac-scans-256.png}
\caption{Time to run a single scan in a SDM with $n=256$, $H=1,000,000$ and $r=103$.
\label{fig:perf-imac-scan-256}}
\end{figure}

\begin{figure}[!htb]
\centering\includegraphics[width=\textwidth]{images02/performance/imac-scans-1000.png}
\caption{Time to run a single scan in a SDM with $n=1,000$, $H=1,000,000$ and $r=451$.
\label{fig:perf-imac-scan-1000}}
\end{figure}

\begin{figure}[!htb]
\centering\includegraphics[width=\textwidth]{images02/performance/imac-scans-10k.png}
\caption{Time to run a single scan in a SDM with $n=10,000$, $H=1,000,000$ and $r=4845$.
\label{fig:perf-imac-scan-10k}}
\end{figure}

\begin{table}[!htb]
\centering
\begin{tabular}{| l | r | r | r |}
    \hline
    & Loops & Scans / second & Scan time (ms) \\ \hline
    Linear scan & 1000 & 81.62 & 12.25 \\
    Thread scan & 5000 & 143.68 & 6.95 \\
    OpenCL scan & 5000 & 238.00 & 4.20 \\ \hline
    \hline
    & Loops & Ops / second & Op. time (ms) \\ \hline
    Thread write & 1000 & 74.92 & 13.34 \\
    Thread single read & 1000 & 96.22 & 10.39 \\
    OpenCL write & 1000 & 126.50 & 7.90 \\
    OpenCL single read & 1000 & 190.20 & 5.25 \\
    \hline
\end{tabular}
\caption{iMac Retina 5K 27-inch 2017 with a 3.8GHz Intel core i5 processor, 8GB DDR4 RAM, and a Radeon Pro 580 8G GPU. Running an SDM with $n=1,000$ bits, $H=1,000,000$, and $r=451$.
\label{tab:perf-imac-1000}}
\end{table}

\begin{table}[!htb]
\centering
\begin{tabular}{| l | r | r | r |}
    \hline
    & Loops & Scans / second & Scan time (ms) \\ \hline
    Linear scan & 1000 & 175.48 & 5.69 \\
    Thread scan & 5000 & 352.63 & 2.83 \\
    OpenCL scan & 5000 & 244.88 & 4.08 \\ \hline
    \hline
    & Loops & Ops / second & Op. time (ms) \\ \hline
    Thread write & 2000 & 304.46 & 3.28 \\
    Thread single read & 2000 & 391.21 & 2.55 \\
    OpenCL write & 2000 & 378.44 & 2.64 \\
    OpenCL single read & 2000 & 466.16 & 2.14 \\
    \hline
\end{tabular}
\caption{iMac Retina 5K 27-inch 2017 with a 3.8GHz Intel core i5 processor, 8GB DDR4 RAM, and a Radeon Pro 580 8G GPU. Running an SDM with $n=256$ bits, $H=1,000,000$, and $r=103$.
\label{tab:perf-imac-256}}
\end{table}

\begin{table}[!htb]
\centering
\begin{tabular}{| l | r | r | r |}
    \hline
    & Loops & Scans / second & Scan time (ms) \\ \hline
    Linear scan & 100 & 8.59 & 116.38 \\
    Thread scan & 500 & 18.66 & 53.56 \\
    OpenCL scan & 1000 & 77.20 & 12.95 \\
    \hline
\end{tabular}
\caption{iMac Retina 5K 27-inch 2017 with a 3.8GHz Intel core i5 processor, 8GB DDR4 RAM, and a Radeon Pro 580 8G GPU. Running an SDM with $n=10,000$ bits, $H=1,000,000$, and $r=4845$.  There is no benchmark for read and write operations because RAM is not enough to allocate the counters --- it would consume 37.25 GB of RAM.
\label{tab:perf-imac-10k}}
\end{table}


% Application
% !TEX root = ../partial-sdm.tex

\section{Results (iv): Supervised classification application}

The supervised classification problem consists of categorize data into groups after seeing some samples from each group. First, it is presented pieces of data with their categories. The algorithm learns from these data, which is known as learning phase. Then, new pieces of data are presented and the algorithm must classify them into the already known groups. It is named supervised because  the algorithm will not create the groups itself. It will learn the groups from during the learning phase, in which the groups have already been defined and the pieces of data have already been classified into them.

Although this problem has already been studied (REF), our intention here is to show that a pure SDM may also be used to classify data. \citet{fan1997genetic} has used SDM to solve a classification problem, recognizing handwriting letters from images, but he used a mix of genetic algorithm with SDM, which is very different from the original SDM described by \cite{Kanerva1988}. Even though his algorithm has classified properly, we were intrigued whether a pure SDM would also classify successfully.

Hence, we have developed a supervised classification algorithm based on a pure SDM as our main memory. Our goal was to classify noisy images into their respective letters (case sensitive) and numbers. For some examples, see Figure \ref{fig-classification-examples}.

\begin{figure}[h]
\centering\includegraphics[width=\textwidth]{./images02/classification/example.png}
\caption{Examples of noisy images with uppercase letters, lowercase letters and numbers.
\label{fig-classification-examples}}
\end{figure}

Each image was mapped into a bitstring in which the bits were set according to the color of each pixel of the image. So, white pixels were equal to bit 0, and black pixels to bit 1. This was a bijective mapping (or one-to-one mapping), i.e., there was only one bitstring for each image, and there was only one image for each bitstring.

We had a total of 62 classification group. For each of them, it was generated a random bitstring. Thus, the groups' bitstrings were orthogonal between any two of them.

During the learning phase, we have generated 100 noisy images for each character. For an example, see the generated images for letter A in Figure \ref{fig-classification-training-A}. Then, we have wrote the classification group bitstring into the bitstring associated to each noisy image, i.e., write(bs\_image, bs\_label).

\begin{figure}[h]
\centering\includegraphics[width=\textwidth]{./images02/classification/trainingA.png}
\caption{100 noisy images generated to train label A.
\label{fig-classification-training-A}}
\end{figure}


\section{Results (iii): Noise filtering application}

\section{Results (v): The possibility of unsupervised reinforcement learning}




\chapter{Results (vii): Information-theoretical write operation}


\begin{figure}[h!]
  \centering
  \subfloat[$w_1(d), d \in \{ 1, 2, ..., n\}.$]{\includegraphics[width=3.2in]{./images02/new-images/Info-theory-a-global.jpeg}}

  \subfloat[$w_1(d)$ for the desired range. ]{\includegraphics[width=3.2in]{./images02/new-images/Info-theory-b-to-500.jpeg}}

  \subfloat[stepwise $\left \lfloor{w_1(d)}\right \rfloor$ for fast integer computation.]{\includegraphics[width=3.2in]{./images02/new-images/Info-theory-c-stepwise.jpeg}}

  \caption{Shannon write operation:  Computing the amount of information of a signal to each hard location in its access radius. (a) entirety of the space; (b) region of interest; (c) Fast integer computation is possible through a stepwise function.}
  \label{fig:info-theory-hypothesis}
\end{figure}



\begin{figure}[h!]
  \centering
  \subfloat[$w_2(d), d \in \{ 1, 2, ..., n\}.$]{\includegraphics[width=3.2in]{./images02/new-images/Info-theory-a-sum.jpeg}}

  \subfloat[$w_2(d)$ for the desired range. ]{\includegraphics[width=3.2in]{./images02/new-images/Info-theory-b-sum-range.jpeg}}

  \subfloat[stepwise $\left \lfloor{w_2(d)}\right \rfloor$ for fast integer computation.]{\includegraphics[width=3.2in]{./images02/new-images/Info-theory-c-sum-stepwise.jpeg}}

  \caption{SOON TO BE DEPRECATED.  Shannon write operation:  Computing the sum of low-likelihood signals. (a) entirety of the space; (b) region of interest; (c) Fast integer computation through a stepwise function. }
  \label{fig:info-theory-sum-hypothesis}
\end{figure}




\begin{figure}[h!]
  \centering
  \subfloat[Kanerva's model]{\includegraphics[width=\linewidth]{./images02/new-images/kanerva-15iter-reads.jpeg}}

  \subfloat[Write process weighted by the amount of information contained in the distance between the written bitstring and each hard location]{\includegraphics[width=\linewidth]{./images02/new-images/info-theory-15iter-reads.jpeg}}


  \caption{(a) and (b) show the behavior of the critical distance under Kanerva's model and the information-theoretic one, respectively.}
  \label{fig:info-theory-experiments}
\end{figure}



\begin{figure}[h!]
  \centering
  \subfloat[$w_2(d), d \in \{ 1, 2, ..., n\}.$]{\includegraphics[width=3.2in]{./images02/new-images/Info-theory-shannon.jpeg}}

  \subfloat[$w_2(d)$ for the desired range. ]{\includegraphics[width=3.2in]{./images02/new-images/Info-theory-shannon-region.jpeg}}

  \subfloat[stepwise $\left \lfloor{w_1(d)}\right \rfloor$ for fast integer computation.]{\includegraphics[width=3.2in]{./images02/new-images/Info-theory-shannon-stepwise.jpeg}}

  \caption{Shannon write operation:  Computing the amount of information of a signal to each hard location in its access radius. (a) entirety of the space; (b) region of interest; (c) Fast integer computation is possible through a stepwise function.}
  \label{fig:info-theory-hypothesis}
\end{figure}



My advisor, Alexandre Linhares, has proposed another write operation: an information-theoretical weighted write. In it, the sum of the counter's value is weighted based on the distance between each hard-location's address and the reading address. The logic behind it is to vary the importance of each hard-location inside the circle.  It is only natural that one encodes an item in closer hard locations with a stronger signal, and a natural candidate for this signal function is the amount of information contained in the distance between the item and each hard location.  Closer hard locations have lower probabilies and therefore should encode more information.



Consider the following. Information Theory \citep{cover2012elements} let us compute the precise amount of information in an event, when given its probability $p$, through the measure of \emph{self-information}:\\

$I(p)= -log_2(p).$ \\

Now, given any two $n$-sized bitstrings, the probability of their Hamming distance being $d$ is given by,


$p(H=d)= {2^{-n} \binom{n}{d} }$ \\

And the probability of it being at most $d$ is \\

$p(H\leq d)= 2^{-n} {\displaystyle\sum_{i=0}^{d}{\binom{n}{i}}}  $, \\

and, consequently,

$p(H\geq n-d)=2^{-n}{\displaystyle\sum_{i=n-d}^{n}{\binom{n}{i}}  }$, \\

$p(d+1 \leq H \leq n-d-1)=2^n - 2^{1-n}{\displaystyle\sum_{i=0}^{d}{\binom{n}{i}}, \forall d<n/2}$. \\

Hence the weighted write would, on each hard location, sum (or subtract) the following:  \\

$w(d) = -\log_2 \left( 2^{-n} \binom{n}{d} \right) = n - \log_2 \binom{n}{d}$, as seen in Figure \ref{fig:info-theory-hypothesis}.

It is easy to interpret this data though a binary tree approach.  How many binary questions would be needed to precisely define a bitstring?

Another possibility would be to use the sum of all distances closer (and less likely) locations within the weighting function $w(d)$,

$w(d) = -\log_2 \left( 2^{-n} \displaystyle\sum_{i=0}^{d}{\binom{n}{i}} \right) = n - \log_2 \displaystyle\sum_{i=0}^{d}{\binom{n}{i}}$. \\

This can be seen in \ref{fig:info-theory-sum-hypothesis}.

The initial results of this \emph{Shannon write} operation can be seen in Figure \ref{fig:info-theory-experiments} and seem promising. It seems that the critical distance increases by a number of bits.  Note that 10 additional bits imply an attractor $2^{10}$ of the size of the original. Another point to keep in mind is that, since the modulus of the vectors are not uniform in this approach, that the shape of the attractor may have asymmetries.



Note, finally, that this is not the first time in which a weighted function has been applied to writing in SDM --- \citet{hely1997new} suggest a rather complex spreading model based on floating point signals in the interval [0.05, 1.0] --- they were, however, only able to test their model with 1,000 hard locations.








\chapter{Conclusion}

Sparse Distributed Memory is a viable model of human memory, yet it does require researchers to (re-)implement a number of parallel algorithms in different architectures.

We propose to provide a new, open-source, cross-platform, highly parallel framework in which researchers may be able to create hypotheses and test them computationally through minimal effort. The framework is well-documented for public release at this time (http://sdm-framework.readthedocs.io), it has already served as the backbone of Chada's Ph.D. thesis. The single-line command ``pip install sdm'' will install the framework on posix-like systems, and single-line commands will let users test the framework, generate some of the figures from Kanerva's theoretical predictions in their own machines, and --- if interested enough ---, test their own theories and improve the framework, and the benchmarks used to evaluate the framework, in open-source fashion. It is our belief that such work is a necessary component towards accelerating research in this promising field.

\section{Future work}

Here are interesting questions that have been considered during this work, but have had to be left for future research.

\subsection{Multiple levels}

Deep neural networks, alphaGo, alphazero



\subsection{i versus l}


\subsection{Magic numbers}

Kanerva suggests, in his book, the use of 1,000 dimensions and 1,000,000 hard locations.  More recently, he suggested the use of 10,000 dimensions, and on personal discussions suggested that this should be a minimum; as he has been concerned in latent semantic analysis and seems to be the proper scale in that application.

Each parameter set choice like this will lead to particular numbers --- many of them emergent---, such as the access radius size, critical distance, and so forth.

One intriguing question here is:  is there a `better' number of dimensions and of hard locations?  If so, can such numbers better studied analitically, or numerically?

How should these parameters be compared?  What are the tradeoffs that should be considered?  What are the `best' benchmarks possible?

\subsection{Classification with context using sequences --- for words instead of only letters}










\subsection{Symmetrical, rapidly accessible, Hard Locations}

A hypercube with n dimensions can be divided by two hypercubes with $n-1$ dimensions. Is there an algorithm that separates the area of each hard-location in such a form that there exists a function mapping each bitstring in $\{0,1\}^n$ to the set of hard locations it `belongs to'?  Though this would break Kanerva's assumption of a randomly yet uniformly distributed set of hard locations --- for a perfectly symmetrical set of hard locations ---, there could be large performance gains if such a mapping function from a bitstring to its corresponding set of nearest hard locations exists.

Consider the hypercube with $n$ dimensions.  We want to select a subset of its vertices with cardinality $2^{20}$ that is symmetrically distributed over the space. Afterwards, $\forall b \in \{ 0,1\} ^n$, we want an algorithm $A$ that yields the particular list of hard locations for $b$ and all hard locations respect the desired properties of the memory.

A reduction from measuring the distance to $2^{20}$ hard locations to a computation of $2^{10}$ hard locations might yield astonishing performance gains, depending, of course, on our optimistic assumptions concerning existence and complexity of such algorithm.  At large scales of computing, the very ability to perform some experiments is a function of sheer performance. The horizon of experiments --- and possibly of knowledge --- expands \emph{as a function of computational demands}. A little more on this in my closing words.

\section{Seeing farther}


Let us revisit, in these concluding thoughts, the emphasis employed over speed of computation.  At first sight, that might seem like a typical objective of efficiency in computer science. But we are not only interested in the computer science effects here --- the ambition is different. More important than this `computer-sciency' goal, i.e., a beautiful, clean, efficient algorithm with the primary effect of enhanced speed, however, is the secondary effect on the sociology of science:  \emph{We can see farther}.

If

We have generated a Docker image, which makes it even easier to explore the framework. After running the container, a Jupyter Notebook is available with sdm-framework and other tools already installed.


All the simulations and graphics generated in this thesis are promptly available to be re-executed and explored by those interested. We invite readers to take a look and explore a little bit.



The overarching intention here is to not only provide a starting point, but to provide a Framework in which SDM research can be conducted.  Consider, for example, having the ability to compare the results of a new (‘forked’) model to the previous `best' (under a particular benchmark set).  For example, some of the benchmarks that we plan to develop in future research is: how fast is convergence through iterative reading?  How large is the attractor of the critical distance?  How well does the system filter noise?  How well does the system work under the supervised learning task?  And other authors may be able to improve this benchmark set themselves, as is usual in open source development.  It is perhaps this facility of ease to build on top of previous work that seems most exciting at this stage.

Consider the misunderstanding concerning the SDM read operation:  Dr Stan Franklin describes Kanerva's read operation in a way that each hard location, at each dimension, provides only a single bit of information to the read operation (instead of Kanerva's full counter).  We have referred to this modified read operation as Chada read (the tale is that my friend \text{\&} colleague, Dr. Daniel de Magalhães Chada, along with Linhares, did not consult Kanerva's book and only discovered the discrepancy in code and ideas a couple of years afterwards).  Having an open, testable, codebase reduces the possibilities of such misunderstandings in the long run.  Indeed, a high-quality codebase seems to have become a scientific community's form of unequivocally standing behind a consensus. For example, the journal Nature analysed the top-100 cited papers in history, to find:

\begin{quote}
... some surprises, not least that it takes a staggering 12,119 citations to rank in the top 100 — and that many of the world’s most famous papers do not make the cut. A few that do, such as the first observation of carbon nanotubes (number 36) are indeed classic discoveries. But the vast majority describe experimental methods or \emph{software that have become essential in their fields. [...] The list reveals just how powerfully research has been affected by computation and the analysis of large data sets.} \\
\hfill --- \citet{van2014top}, emphasis mine.
\end{quote}

It is no coincidence that scientific journals such as BMC Neuroscience, or the Journal of Machine Learning Research have specific sections on open-source software. The journal Neurocomputing states, bluntly: “software is scientific method by machine”.

Of course, for the skeptical reader who may consider software a less worthy pursuit, there is also new work here.  The mathematics of the model has been shown to be correct numerically (with a single, small, anomaly); we have shown how to execute unsupervised learning with nothing besides operations original to the SDM; we have studied the generalized Murilo read; we have seen noise filtering; the death of neurons; how information-theory may be of use; and finally, we have reproduced numerous of the original propositions put forth by Kanerva.  The emphasis might have been on \emph{breadth of topics}, in detriment of depth here or there.  But this is due to our research group's enthusiam for the topic; we do indeed believe that SDM is --- if not correct --- extremely close to a full scientific understanding of human long-term memory.  If so, it is such a monumental achievement that we want readers to be able to see all of what we see and imagine the vastness of possibilities.  The work on, say, reinforcement learning, is most definitely not the definitive work we will see on the subject, but a challenge left for readers to contemplate. Ralph Waldo Emerson once said \emph{Do not go where the path may lead. Go, instead, where there is no path, and leave a trail.}  Dr. Kanerva has left the trail.  It is my job to pave it and to throw light at it and to try to deliver an easier pathway for the next generation.  Some essays completely shut the door close at the end; this one intends to leave it wide open. As the reader might have noticed, this final section does not read as an analysis of the work done; it reads, instead, as a \emph{desiderata}, a prologue, a yearning for others to join me in imagining the shape of things to come.










% TODO Include in the to do list simulations based on the suggestion of Murilinho.

\chapter{Appendix}
%\section{Generating Kanerva's table 7.3}

%\includepdf[
%  pages=-,
%  pagecommand={\pagestyle{headings}},
%  %addtotoc={1,section,1,Quilling Shapes,sec:shapes}
%]{./Chapter02-SDM/Kanerva-Table-7.3/Kanerva-Table-7_3.pdf}

\chapter*{List of jupyter notebooks}
