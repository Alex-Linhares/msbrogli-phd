\chapter{Conclusion}
\bigskip

\begin{flushright}{\slshape
	Software is eating the World.} \\ \medskip
    --- Mark Andreesen \citep{andreesen:2011}
\end{flushright}
\bigskip
\bigskip

\noindent Modern management and high technology interact in multiple, profound, ways.  Software, in particular, seems to have an immense power of entering arenas which seemed, at some point, to require either specific hardware or the skill of humans.  One of the members of this thesis committee, Dr. Nichols, will participate through teleconferencing over the open web, with no use of hardware specific for the task.  The corporate biography of Tonny Martins, President of IBM Brasil, mentions his successes with blockchain, AI and cognitive technology… as an executive, not as a research scientist or specialized engineer (https://www-03.ibm.com/press/br/pt/biography/53561.wss). Professor Andrew Ng tells students at Stanford’s Graduate School of Business that ``AI is the new electricity'' \citet{AndrewNgAI}, as his way to emphasize the potential transformational power of the technology.  It is not impossible that a purely digital form of money may exist.  It is not impossible that machines may become intelligent.  Moreover, it is not impossible that these two processes may have already begun.  

It is worthwhile, in this concluding section, to reflect on some ideas on what this thesis is and what it is that we, as computational management scientists, can obtain from this sort of study.  Clemenceau once said that ``war is too important to be left to the generals''.  I believe it is not far-fetched to state that technology has become too important to be isolated to the realm of computer science, or engineering, or applied mathematics, or any single discipline.  The emergence of scientific journals with names such as Computational Management Science; INFORMS Journal on Computing; Ledger; Computational Statistics; ACM Transactions on Economics and Computation, and so forth, show that there are growing communities deeply interested in the intersection of business and the computer sciences.  To whom, for instance, does the OpenAI project belong?  To computing or to business?  Recall that the project was created as a risk-management strategy against the far-fetched, but not impossible, possibility of having machines yielding too much power.  What about corporations like Uber?  AirBnB?  Imagine a new method that increases profits by 50\% at a tech company.  Should this method, if implementable as an algorithm (like PageRank \citet{brin1998anatomy}) or a data structure (like a blockchain) be discussed in conferences of `computer science' or `business'? It seems quite arbitrary to name a single group, as a whole new ecosystem seems to be emerging within those two. That is why this thesis is computational and why it is business.  This work explores topics that seem, on the surface, to belong to computer science, but their applicability and impact to businesses seem too large to be delegated away, something ``for the nerds in the fifth floor''.  As technical decisions become central to the organization of man's life, the technician becomes the visionary, the innovator, the decision-making arbiter, sometimes the billionaire.  

We have started this study with two possible forms of organization of a purely digital money system; a blockchain and a directed acyclic graph; we then moved to an analysis of the adoption of new technologies.  

The possibility that there will be some form of purely digital money has become very real.  Consider, just as a matter of comparison, Brazil's most important company: Petrobrás.  As of this writing, the ``market cap'' of Ethereum exceeds that of Petrobras by ten billion dollars, while Bitcoin's is valued at more than double of Petrobras (195B usd vs 83B usd).  These technologies should, at a minimum, be taken seriously. 



%P.5. CryptoPonzi #2. Dag/Tangle Results. Some speculative possibilities can be mentioned here:  



We have explored Kanerva’s Sparse Distributed Memory.   In AI, SDM seems to be a particularly interesting area for study.  The model plausibly reflects a number of well known aspects of psychology and neuroscience.  For example, neurons can easily compute the address decoding scheme of the system.  Neurons are fragile and may be lost, whereas the information remains preserved, due to the distributed character of the model.  The ``tip-of-the-tongue'' behavior emerges naturally, and so does Miller’s magic number.

There are three contributions made on SDM:  First, I have illuminated a discrepancy between Kanerva’s theoretical model and the real system dynamics; Also, we have seen that pattern classification through supervised learning is possible without presuming any new SDM mechanism. This is in contrast with the literature, that presumes additional mechanisms, like genetic algorithms, to account for supervised learning.  Finally, we now have a tested open-source framework that offers parallelism and can become a de-facto standard in SDM research.  The framework (i) carefully reproduces crucial figures from Kanerva’s theoretical book; (ii) shows how noise filtering and (iii) supervised learning can be done, and, through the use of (iv) Jupyter Notebooks, enables the reader to easily reproduce all the results on their own machines. This respects all constraints posed by Robert M. French in his article on Computational Modeling in Cognitive Science: A Manifesto for Change \citep{TOPS:TOPS1206}.

The ability to rapidly reproduce results, and to build on prior work, is, I believe, fundamental to modern science.  Consider, for instance, the groundbreaking successes in the arena of deep learning.  Having standard computer libraries to work with has brought together a community, which reinforces the system, as users also gradually improve these libraries. It may be possible to achieve new results with multiple layers of a SDM, yet, having to start development from scratch takes a large opportunity cost from most scientists --- especially those who are less concentrated on the computer science aspects, but still would be able to contribute meaningfully.  

Finally, we have studied how variations of the Bass Model may reflect systems or technologies that may wither in time. Though some innovations, such as the radio, have gained widespread use in a sustainable form... One may want to review the Bass model when one is concerned with rapidly-evolving technological ecosystems.  Hardly anyone remembers the names AskJeeves, World Wide Web Worm, Lycos, WebCrawler, or AltaVista, early web search engines; later replaced, in the market and by the market, by the almost unnoticed url http://google.stanford.edu \citep{brin1998anatomy}.


Another possibility would be to compare the proposed model with a computation of the momentum of Metcalfe’s law in between competitors.  As the reader may remember, Metcalfe’s law states that the value of a network grows $O(n^2)$ with $n$ being the number of network nodes.   If the proposed model and Metcalfe’s network effects reflect reality, then there could be an integrated mathematical model that explains and represents both Metcalfe’s law and the variation of the Bass model presented herein.  

With this, I submit this thesis in the hope that all readers, present and future, may find the aforementioned studies as useful, genuine, and legitimate contributions to the thriving field of Computational Management Science.
