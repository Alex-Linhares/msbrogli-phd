\chapter{Conclusion}
\bigskip

\begin{flushright}{\slshape
	Software is eating the World.} \\ \medskip
    --- Mark Andreesen \citep{andreesen:2011}
\end{flushright}
\bigskip
\bigskip

\noindent Modern management and high technology interact in multiple, profound, ways.  Software, in particular, seems to have an immense power of entering arenas which seemed, at some point, to require specific hardware or the skill of humans.  One of the members of this thesis committee, Dr. Nichols, will participate through teleconferencing over the open web, with no use of hardware specific for the task.  The corporate biography of Tonny Martins, President of IBM Brasil, mentions his successes with blockchain, AI and cognitive technology... as an executive, not as a research scientist or specialized engineer (https://www-03.ibm.com/press/br/pt/biography/53561.wss). Professor Andrew Ng tells students at Stanford’s Graduate School of Business that “AI is the new electricity”, as his way to emphasize the potential transformational power of the technology.  It is not impossible that a purely digital form of money may exist.  It is not impossible that machines may become intelligent.  Moreover, it is not impossible that these two processes may be already beginning.  

@video{AndrewNgAI, 
  title={Artificial Intelligence is the New Electricity},
  author={Andrew Ng},
  journal={Stanford Graduate School of Business },
  year={2017},
  url={https://www.youtube.com/watch?v=21EiKfQYZXc}
}

Clemenceau once said that “war is too important to be left to the generals”.  I believe it is not far-fetched to state that technology has become too important to be isolated to the realm of computer science, or engineering, or applied mathematics, or any single discipline.  The emergence of scientific journals with names such as Computational Management Science; INFORMS Journal on Computing; Ledger; Computational Statistics; ACM Transactions on Economics and Computation, and so forth, show that there are growing communities deeply interested in the intersection of business and the computer sciences.  To whom, for instance, does the OpenAI project belong?  To computing or to business?  It seems quite arbitrary to name a single group. What about corporations like Uber?  AirBnB?  eBay?  Amazon? That is why this thesis is computational and why it is business.  This work explores topics that seem, on the surface, to belong to computer science, but their applicability and impact to businesses seem too large to be delegated away.  

We have started this study with two possible forms of organization of a purely digital money system; a blockchain and a directed acyclic graph.  Are these cryptocurrencies just a new financial asset, or are they scalable to the unbanked?

Bitcoin

P.5. CryptoPonzi \#2. Dag/Tangle Results. Some speculative possibilities can be mentioned here:  



P.6. We have explored Kanerva’s Sparse Distributed Memory.   


P. 7. Pattern classification through SDM; Supervised learning 


P.8. Representation and Deep learning with SDM.  


P. 9. Finally, we have studied how variations of the Bass Model may reflect systems or technologies that may wither in time. 

Another possibility would be to compare the proposed model with a computation of the momentum of Metcalfe’s law in between competitors.  As the reader may remember, Metcalfe’s law states that the value of a network grows $O(n^2)$ with $n$ being the number of network nodes (Note that some have argued that the value growth rate is actually $O(n log n)$ CITATION CITATION CITATION). 

If the proposed model and network effects reflect reality, then there could be an integrated mathematical model that explains and represents both Metcalfe’s law and the variation of the Bass model presented herein.  

P.10.  Message / future is open / thesis is closed.  