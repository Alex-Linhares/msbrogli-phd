%*******************************************************
% Abstract
%*******************************************************
%\renewcommand{\abstractname}{Abstract}
\pdfbookmark[1]{Abstract}{Abstract}
\begingroup
\let\clearpage\relax
\let\cleardoublepage\relax
\let\cleardoublepage\relax

\chapter*{Abstract}

This thesis presents three specific, self-contained, scientific papers in the Computational Management Science area.
Modern management and high technology interact in multiple, profound, ways. Professor Andrew Ng tells students at Stanford’s Graduate School of Business that ``AI is the new electricity'', as his hyperbolic way to emphasize the potential transformational power of the technology.

The first paper is inspired by the possibility that there will be some form of purely digital money and studies distributed ledgers, proposing and analyzing Hathor, an alternative architecture towards a scalable cryptocurrency.

The second paper may be a crucial item in understanding human decision making, perhaps, bringing us a formal model of recognition-primed decision. Lying at the intersection of cognitive psychology, computer science, neuroscience, and artificial intelligence, it presents an open-source, cross-platform, and highly parallel framework of the Sparse Distributed Memory and analyzes the dynamics of the memory with some applications.

Last but not least, the third paper lies at the intersection of marketing, diffusion of technological innovation, and modeling, extending the famous Bass model to account for users who, after adopting the innovation for a while, decide to reject it later on.

%\vfill

\chapter*{Resumo}
A presente tese é formada por três trabalhos científicos na área de Management Science Computacional. A gestão moderna e a alta tecnologia interagem em múltiplas e profundas formas. O professor Andre Ng diz aos seus estudantes na Escola de Negócios de Stanford que ``Inteligência Artificial é a nova eletricidade'', como sua forma hiperbólica de enfatizar o potencial transformador da tecnologia.

O primeiro trabalho é inspirado na possibilidade de que haverá alguma forma de dinheiro digital e estuda ledger distribuídas, propondo e analisando o Hathor, uma arquitetura alternativa para criptomoedas escaláveis.

O segundo trabalho pode ser um item crucial no entendimento de tomadas de decisão, nos trazendo um modelo formal de \emph{recognition-primed decisions}. Situada na intersecção entre psicologia cognitiva, ciência da computação, neuro-ciência e inteligência artifical, ele apresenta um framework open-source, multi-plataforma e altamente paralelo da Sparse Distributed Memory e analisa a dinâmica da memória e algumas aplicações.

O terceiro e último trabalho se situa na intersecção entre marketing, difusão de inovação tecnologica e modelagem, extendendo o famoso modelo de Bass para levar em consideração usuário que, após adotar a tecnologia por um tempo, decidiram rejeitá-la.

\endgroup

%\vfill
